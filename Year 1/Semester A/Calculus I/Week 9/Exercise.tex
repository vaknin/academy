% Preamble
\documentclass[a4paper, 12pt, leqno]{article}
\usepackage[margin=1in]{geometry} % Set margin
\usepackage{amssymb,amsmath,amsthm, amsfonts} % Math libraries
\usepackage{pdfpages} % Insert PDF pages

% Hebrew support
\usepackage[utf8x]{inputenc}
\usepackage{culmus}
\usepackage[english,hebrew]{babel}
\selectlanguage{hebrew}

% Custom commands
\newcommand{\sub}[1]{\subsection{\underline{#1}}}
\newcommand{\subsub}[1]{\subsubsection{\underline{#1}}}
\newcommand{\F}{\ensuremath{\mathbb{F}}}
\newcommand{\N}{\ensuremath{\mathbb{N}}}
\newcommand{\RR}{\ensuremath{\mathbb{R}}}
\newcommand{\Onef}{\ensuremath{1_{\F}}}
\newcommand{\Zerof}{\ensuremath{0_{\F}}}
\newcommand{\eqbcuz}[1]{\text{~$\stackrel{(#1)}{=}$~}}
\newcommand{\eq}[1]{\begin{align*}#1\end{align*}}
\newcommand{\eqn}[1]{\begin{align}#1\end{align}}
\newcommand{\set}[1]{\big{\{} #1 \big{\}}}
\newcommand{\bigset}[1]{\bigg{\{} #1 \bigg{\}}}
\newcommand{\limn}{\lim_{n\to\infty}}
\newcommand{\sumn}{\sum^{n}_{i=1}}

\renewcommand{\qed}{\hfill\(\qedsymbol\)}
\renewcommand{\leq}{\leqslant}
\renewcommand{\geq}{\geqslant}

% Begin Document %
\begin{document}

% Title Page
\begin{titlepage}
    \begin{center}
        \vspace*{4cm}
    
        {\fontsize{32pt}{32pt}\selectfont \textbf{אינפי 1}}
        
        \vspace{0.4cm}
        
        {\LARGE
        תרגיל 9}
    
        \vfill
            
        {
            \Large\textbf{אביב וקנין}
            \\
            \selectlanguage{english}316017128
        }
    \end{center}
\end{titlepage}

% 1
\section{}
\sub{}
הטור לא מתכנס.\\
נראה זאת לפי מבחן ההשוואה.\\
ידוע כי מתקיים לכל $n$ טבעי:
\eq{
    \frac{1}{n^{1+\frac{1}{n}}}\leq\frac{1}{n}
}
ולפי מבחן ההשוואה, מכיוון שהטור ההרמוני אינו מתכנס, גם הטור הזה אינו מתכנס.
\qed
\sub{}
הטור אכן מתכנס, נראה זאת לפי מבחן השורש:
\eq{
    \sqrt[n]{(\frac{n}{n+1})^{n^2}}=(\frac{n}{n+1})^n=\frac{1}{(1+\frac{1}{n})^n}
}
כעת, נחשב את הגבול:
\eq{
    \limn\frac{1}{(1+\frac{1}{n})^n}=\frac{1}{e}\eqsim0.3678<1
}
לכן, לפי מבחן השורש, הטור מתכנס.
\qed
\setcounter{subsection}{3}
\sub{}
הטור מתכנס, נראה זאת לפי מבחן המנה.
\eq{
    \limn{\frac{a_{n+1}}{a_n}}=\limn{\frac{(n+1)^22^n}{n^22^{n+1}}}=\limn\frac{(n+1)^2}{2n^2}=\frac{1}{2}
}
מכיוון שגבול המנה קטן מ-$1$, הטור אכן מתכנס.
\sub{}
נראה כי הטור אינו מתכנס, בעזרת מבחן המנה.
\eq{
    \frac{a_{n+1}}{a_n}=\frac{3^{n+1}(n+1)!n^n}{3^nn!(n+1)^{n+1}}=\frac{3n^n}{(n+1)^n}=3(\frac{n}{n+1})^n=3(\frac{1}{1+\frac{1}{n}})^n=\frac{3}{(1+\frac{1}{n})^n}
}
מכיוון שידוע לנו כי:
\eq{
    \limn\frac{1}{1+\frac{1}{n}}=e
}
נוכל להסיק כי הגבול הוא:
\eq{
    \limn\frac{3}{(1+\frac{1}{n})^n}=\frac{3}{e}\eqsim1.1036>1
}
לכן, נוכל להסיק לפי מבחן המנה כי הטור אינו מתכנס.
\qed
\setcounter{subsection}{6}
\sub{}
הטור מתכנס, נראה זאת לפי מבחן השורש.\\
כלומר, עלינו להראות כי:
\eq{
    \limn{\sqrt[n]{n}-1}<1
}
ואכן:
\eq{
    \limn{\sqrt[n]{n}-1}=\limn\sqrt[n]{n}-\limn1=1-1=0
}
כפי שרצינו, הגבול קטן מ-$0$ ולכן הטור מתכנס לפי מבחן השורש.
\sub{}
נראה כי הטור אכן מתכנס, בעזרת מבחן המנה.
\eq{
    \frac{a_{n+1}}{a_n}=\frac{2^{n+1}(n+1)!n^n}{2^nn!(n+1)^{n+1}}=\frac{2n^n}{(n+1)^n}=2(\frac{n}{n+1})^n=2(\frac{1}{1+\frac{1}{n}})^n=\frac{2}{(1+\frac{1}{n})^n}
}
מכיוון שידוע לנו כי:
\eq{
    \limn\frac{1}{1+\frac{1}{n}}=e
}
נוכל להסיק כי הגבול הוא:
\eq{
    \limn\frac{2}{(1+\frac{1}{n})^n}=\frac{2}{e}\eqsim0.7357<1
}
לכן, נוכל להסיק לפי מבחן המנה כי הטור מתכנס.
\qed
\pagebreak

% 2
\section{}
\sub{}
\subsub{$\impliedby$}
נניח כי הטור של $a_n$ מתכנס, ולכן, הוא מקיים את קריטריון קושי להתכנסות טורים, ובפרט, לכל $\varepsilon>0$ קיים $N\in\N$ כך שלכל $k>t>N$ מתקיים:
\eq{
    |\sum_{i=t}^k{a_i}|<\varepsilon
}
כעת, בכדי להראות שהטור של $a_{n+m}$ מתכנס אף הוא, נראה כי לכל $\varepsilon>0$ קיים $N'\in\N$ שנבחר כך:
\eq{
    N'=\max(0,N-m)
}
כאשר ה-$N$ הוא אותו ה-$N$ המתאים לטור $a_n$, כך שלכל $k'>t'>N'$ מתקיים:
\eq{
    |\sum_{i=t'}^{k'}{a_i}|<\varepsilon
}
כלומר, לפי קריטריון קושי להתכנסות טורים, הסדרה $a_{n+m}$ ניתנת לסכימה.
\qed
\subsub{$\implies$}
נניח כי הסדרה $a_{n+m}$ ניתנת לסכימה, לכן, היא מקיימת את קריטריון קושי, ובפרט מתקיים כי לכל $\varepsilon>0$ קיים $N\in\N$ כך שלכל $k>t>N$ מתקיים:
\eq{
    |\sum_{i=t}^{k}{a_{i+m}}|<\varepsilon
}
מכיוון שהסדרה $a_{n+m}$ היא תת-סדרה של $a_n$, הרי האינדקס ה-$N$'י קיים בה, אולם בהזזה מסוימת של $m$ אינדקסים ימינה.\\
כלומר, אם נבחר את $N'$ להיות:
\eq{
    N'=N+m
}
נראה כי לכל $\varepsilon>0$ קיים $N'\in\N$ כך שלכל $k'>t'>N'$ מתקיים:
\eq{
    |\sum_{i=t'}^{k'}{a_{i}}|<\varepsilon
}
כלומר, לפי קריטריון קושי להתכנסות טורים, הסדרה $a_{n}$ ניתנת לסכימה.
\qed
\sub{}
הראינו בסעיף א' כי אם הטור של $a_n$ מתכנס, אזי גם הטור של הסדרה $a_{n+m}$ מתכנס.\\
לכן, לפי הגדרת הסכום, אם נחבר את $m$ האיברים הראשונים לסכום האינסופי של $a_{n+m}$, נקבל בדיוק את $a_n$.
\pagebreak

% 3
\section{}
\sub{}
לא הצלחתי
\sub{}
נראה תחילה כי הסדרה הבאה ניתנת לסכימה:
\eq{
    \frac{1}{\sqrt{n^2+2n}}
}
לפי מבחן המנה:
\eq{
    \limn{\frac
    {\frac{1}{\sqrt{(n+1)^2+2(n+1)}}}
    {\frac{1}{\sqrt{n^2+2n}}}}
    =\limn\frac{\sqrt{n^2+2n}}{\sqrt{(n+1)^2+2(n+1)}}<1
}
קל לראות כי המכנה גדול מהמונה, ולכן לפי מבחן המנה - הסדרה מתכנסת.\\
כעת, לפי משפט לייבניץ, אנו יכולים להסיק כי גם טור לייבניץ של הסדרה מתכנס.\\
מכיוון שהטור מתכנס, אנו יודעים כי לכל $\varepsilon>0$ קיים $N\in\N$ כך שלכל $n>N$ מתקיים:
\eq{
    |S_n-L|<\varepsilon
}
לפי סעיף א', אנו יודעים כי מתקיים:
\eq{
    |S_n-L|\leq|a_{n+1}|=\frac{1}{\sqrt{(n+1)^2+2(n+1)}}\leq\frac{1}{n}
}
לכן, אם נבחר את $N\geq\frac{1}{\varepsilon}$ התנאי יתקיים בהכרח.\\
עבור $\varepsilon=10^{-4}$ נצטרך לבחור את:
\eq{
    N\geq\frac{1}{\varepsilon}=\frac{1}{10^{-4}}=10000
}
לכן, דוגמה לבחירה תהיה $N=10000$.
\qed

% 4
\pagebreak
\section{}
\sub{}
הטענה נכונה.\\
מכיוון שידוע לנו כי:
\eq{
    \limn\frac{a_n}{b_n}=0
}
נוכל להסיק כי לכל $\varepsilon>0$ ובפרט ל-$1$ קיים $N\in\N$ כך שלכל $n>N$ מתקיים:
\eq{
    |\frac{a_n}{b_n}-0|=\frac{a_n}{b_n}<1
}
נכפול את שני אגפי אי-השווין ב-$b_n$ ונקבל כי החל ממקום מסוים מתקיים:
\eq{
    a_n<b_n
}
לפי מבחן ההשוואה, מכיוון ש-$b_n$ ניתנת לסכימה, אזי גם $a_n$ ניתנת לסכימה.
\qed
\sub{}
הטענה אינה נכונה.\\
ניקח את:
\eq{
    a_n&=\frac{n^2}{2^n}\\
    b_n&=42
}
שתי הסדרות אכן מקיימות את תנאי השאלה, ואכן הטור של $a_n$ מתכנס )לפי שאלה 1ד'(, אולם קל לראות כי הסדרה הקבועה $b_n$ אינה ניתנת לסכימה.
\qed\pagebreak

% 5
\section{}
\sub{}
הטענה אינה נכונה עבור שני סוגי הטורים, ניקח את:
\eq{
    a_n=\frac{1}{n}
}
לכן:
\eq{
    a^{2}_{n}=\frac{1}{n^2}
}
הראינו בהרצאה כי $a_n$ לא ניתנת לסכימה, ואילו $a^{2}_{n}$ אכן ניתנת לסכימה.
\qed
\sub{}
\subsub{טורים אי-שליליים}
הטענה נכונה.\\
ניקח סדרה $a_n$ כלשהי אשר ניתנת לסכימה.\\
לכן, לפי ההגדרה, אם נבחר $\varepsilon=1$, אז קיים $N\in\N$ כך שלכל $n>N$ מתקיים:
\eq{
    |a_n|=a_n<1
}
לכן, מכיוון ש-$a_n<1$ החל מאותו $N$ מתקיים:
\eq{
    |a^{2}_{n}|=a^{2}_{n}\leq{a_n}
}
וזאת מכיוון שכל ריבוע של מספר בין 0 ל-1 קטן או שווה למספר עצמו.\\
לכן, נוכל כעת להפעיל את מבחן ההשוואה, ולהסיק כי מכיוון ש$a_n$ ניתנת לסכימה, אזי גם $a^{2}_{n}$.
\qed
\subsub{טורים כלליים}
הטענה אינה נכונה.\\
ראשית, נבין כי הסדרה $b_n=\frac{1}{\sqrt{n}}$ היא סידרה יורדת של מספרים אי-שליליים, המתכנסת לאפס.\\
כעת, לפי משפט לייבניץ ניתן להסיק כי הסדרה הבאה ניתנת לסכימה:
\eq{
    a_n=\frac{(-1)^{n+1}}{\sqrt{n}}
}
כעת, נחשב את $a^{2}_{n}$-
\eq{
    a^{2}_{n}=\frac{1}{n}
}
אולם, אנו יודעים כי $\frac{1}{n}$ לא ניתנת לסכימה.\\
לכן, אנו מבינים כי הטענה אינה נכונה.
\sub{}
\subsub{טורים אי-שליליים}
עבור טורים אי-שליליים, $|a_n|=a_n$ ולכן בדומה לסעיף ב', הטענה נכונה.
\subsub{טורים כלליים}

% End
\end{document}