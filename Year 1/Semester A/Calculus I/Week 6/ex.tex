% Preamble
\documentclass[a4paper, 12pt, leqno]{article}
\usepackage[margin=1in]{geometry} % Set margin
\usepackage{amssymb,amsmath,amsthm, amsfonts} % Math libraries

% Hebrew support
\usepackage[utf8x]{inputenc}
\usepackage{culmus}
\usepackage[english,hebrew]{babel}
\selectlanguage{english}

% Custom commands
\newcommand{\sub}[1]{\subsection{\underline{#1}}}
\newcommand{\subsub}[1]{\subsubsection{\underline{#1}}}
\newcommand{\F}{\ensuremath{\mathbb{F}}}
\newcommand{\N}{\ensuremath{\mathbb{N}}}
\newcommand{\Onef}{\ensuremath{1_{\F}}}
\newcommand{\Zerof}{\ensuremath{0_{\F}}}
\newcommand{\eqbcuz}[1]{\text{~$\stackrel{(#1)}{=}$~}}
\newcommand{\eq}[1]{\begin{align*}#1\end{align*}}
\newcommand{\eqn}[1]{\begin{align}#1\end{align}}
\newcommand{\set}[1]{\big{\{} #1 \big{\}}}
\newcommand{\bigset}[1]{\bigg{\{} #1 \bigg{\}}}
\renewcommand{\qed}{\hfill\(\qedsymbol\)}
\renewcommand{\leq}{\leqslant}
\renewcommand{\geq}{\geqslant}
\newcommand{\limn}{\lim_{n\to\infty}}

% Begin Document %
\begin{document}

% Title Page
\begin{titlepage}
    \begin{center}
        \vspace*{4cm}
    
        {\fontsize{35pt}{35pt}\selectfont \textbf{אינפי 1}}
        
        \vspace{0.4cm}
        
        {\LARGEתרגיל 6}
    
        \vfill
            
        {\Large\textbf{אביב וקנין}\\821710613}
    \end{center}
\end{titlepage}

% 1.2
\section{חשבו את הגבולות של הסדרות הבאות. מותר להשתמש באריתמטיקה של גבולות ומשפט הסנדוויץ.}
\setcounter{subsection}{1}
\sub{}

\eq{
    a_n=\sum^{n}_{k=1}\frac{1}{\sqrt{n^2+k}}
}
נחשב את גבול הסדרה באמצעות משפט הסנדוויץ'.\\
כלומר,עבור $a_n$ ו-$b_n$ מסוימים אשר קל למצוא את גבולם, נרצה להראות כי:
\eq{
    \forall{n}\in\N~~b_n\leq{a_n}\leq{c_n}
}
ומכך, נוכל להסיק כי:
\eq{
    \limn(a_n)=\limn(b_n)=\limn(c_n)
}
ראשית, קל לראות כי אם נגדיר את $b_n$ ואת $c_n$ כך:
\eq{
    b_n&=\frac{1}{n^2}\\
    c_n&=\frac{1}{n}\\
    \forall{n}\in\N~~b_n&\leq{a_n}\leq{c_n}
}
את גבולות סדרות $b_n$ ו-$c_n$ קל לחשב:
\eq{
    &\limn(b_n)=\limn(\frac{1}{n^2})=0\\
    &\limn(c_n)=\limn(\frac{1}{n})=0
}
ולכן, לפי משפט הסנדוויץ':
\eq{
    \limn(a_n)=\limn(b_n)=\limn(c_n)=0
}
\qed\pagebreak
% 1.4
\setcounter{subsection}{3}
\sub{}
\eq{
    a_n=\sqrt[n]{x^n+y^n}
}
מכיוון ש-$0\leq{x}\leq{y}$, ניתן לראות כי:
\eq{
    &\sqrt[n]{y^n}\leq\sqrt[n]{x^n+y^n}\leq \sqrt[n]{y^n+y^n}\\
    &y\leq\sqrt[n]{x^n+y^n}\leq{y}\sqrt[n]{2}
}
כלומר, אם נוכיח כי הגבול של $y$ והגבול של ${y}\sqrt[n]{2}$ שווים, נוכל להניח כי זהו אמנם הגבול של $a_n$.
כעת, נחשב את הגבולות משני צדי $a_n$:
\eq{
    \limn(y)&=y\\
    \limn(\sqrt[n]{2})&=1\\
    \limn(y\sqrt[n]{2})
    =\limn(y)\cdot\limn(\sqrt[n]{2})
    &=y
}
לכן, לפי כלל הסנדוויץ' נוכל להסיק:
\eq{
    \limn(a_n)=y
}
% 1.5
\setcounter{subsection}{4}
\sub{}
\eq{
    a_n=(0.99999+\frac{1}{n})^n
}
לפי אריתמטיקה של גבולות:
\eqn{
    \limn(0.99999)&=0.99999\nonumber\\
    \limn(\frac{1}{n})&=0\nonumber\\
    \limn(0.99999+\frac{1}{n})&=0.99999
}
ובנוסף:
\eqn{
    \limn((0.99999+\frac{1}{n})^n)=(\limn(0.99999+\frac{1}{n}))^n
}
לכן, לפי )1( ו-)2(:
\eq{
    \limn((0.99999+\frac{1}{n})^n)=(0.99999)^n=0
}
\qed\pagebreak

% 2
\setcounter{equation}{0}
\section{הוכיחו כי הסדרות מתכנסות}
נתון לנו כי:
\eq{
    b_n\leq{c_n}
}
לכן:
\eqn{
    b_n-a_n\leq{c_n-a_n}
}
בנוסף, נתון לנו כי:
\eq{
    a_n\leq b_n
}
ומכאן נובע:
\eqn{
    0\leq b_n-a_n
}
לכן, לפי )1( ו-)2(:
\eqn{
    0\leq b_n-a_n\leq{c_n-a_n}
}
בנוסף, אנו יודעים כי:
\eqn{
    \limn(0)&=0\\
    \limn(c_n-a_n)&=0
}
לכן, לפי כלל הסנדוויץ' על )3(, )4( ו-)5( נוכל להסיק כי:
\eqn{
    \limn(b_n-a_n)=0
}
ידוע לנו כי:
\eqn{
    \limn(b_n)=L
}
לפי אריתמטיקה של גבולות על )6( ו-)7(:
\eqn{
    \limn(b_n-(b_n-a_n))&=L-0\nonumber\\
    \limn(a_n)&=L
}
כעת, לפי )5( ו-)8(, נוכל להסיק לפי הלמה של קנטור:
\eq{
    \limn(a_n)=\limn(c_n)=L
}
\qed\pagebreak

% 3.1
\section{}
\sub{הוכיחו כי הסדרה $\sqrt[n]{a_n}$ מתכנסת ל-1.}
\setcounter{equation}{0}
ידוע לנו כי:
\eq{
    0<\alpha<a_n<\beta
}
לכן, ניתן להסיק:
\eqn{
    \sqrt[n]{\alpha}<\sqrt[n]{a_n}<\sqrt[n]{\beta}
}
הוכח כי:
\eqn{
    \forall{a}>0~~\limn(\sqrt[n]{a})=1
}
לכן, לפי )2(:
\eqn{
    \limn(\sqrt[n]{\alpha})&=1\\
    \limn(\sqrt[n]{\beta})&=1
}
לכן, אם נפעיל את כלל הסנדוויץ' על )1(, )3( ו-)4(, נקבל:
\eq{
    \limn(\sqrt[n]{a_n})=1
}
\qed
% 3.2
\sub{הוכיחו כי הסדרה $\sqrt[n]{a_n}$ מתכנסת ל-1.}
\setcounter{equation}{0}
ידוע לנו כי הסדרה מתכנסת ל-$L$ חיובי כלשהו, כלומר:
\eqn{
    \limn{a_n}=L>0
}
הוכחנו בכיתה, כי:
\eqn{
    \forall{a}>0~~\limn{\sqrt[n]{a}}=1
}
לכן, ע"י שילוב )1( ו-)2( נקבל:
\eq{
    \limn{\sqrt[n]{a_n}}=\limn{\sqrt[n]{L}}=1
}
\qed\pagebreak

% 4.1
\setcounter{equation}{0}
\section{חשבו את הגבולות של הסדרות הבאות במובן \textbf{הרחב} ישירות מההגדרה}
\sub{}
\eq{
    a_n=\frac{n^3-100}{n^2+20n}
}
בכדי להוכיח שהסדרה שואפת לאינסוף, עלינו להוכיח כי:
\eq{
    \forall{M>0}~~\exists{N\in\N}~~\forall{n>N}~~a_n>M
}
נבחר N טבעי אשר מקיים:
\eq{
    N>M+20
}
לכן, לכל $n>N$ מתקיים:
\eq{
    a_n>n-20>N-20>(M+20)-20=M
}
לכן, לפי הגדרת הגבול במובן הרחב:
\eq{
    \limn(a_n)=\infty
}
\qed
\setcounter{subsection}{2}
\sub{}
\eq{
    a_n=\lfloor\sqrt{n}\rfloor
}
בכדי להוכיח שהסדרה שואפת לאינסוף, עלינו להוכיח כי:
\eq{
    \forall{M>0}~~\exists{N\in\N}~~\forall{n>N}~~a_n>M
}
נתחיל מהפסוק הבא:
\eq{
    \lfloor\sqrt{n}\rfloor>\sqrt{n}-1
}
הפסוק נכון מכיוון שפונקציית הרצפה תחזיר מספר שמרחקו מערך הביטוי המקורי תמיד יהיה קטן מ-1.\\
לכן, כאשר אנו מחסירים 1 מערך הביטוי, $a_n$ תמיד יהיה גדול יותר.\\
לכן, נבחר:
\eq{
    N\geq (M+1)^2
}
לכן, לכל $n>N$ מתקיים:
\eq{
    a_n>\sqrt{n}-1>\sqrt{N}-1>M
}
לכן, לפי הגדרת הגבול במובן הרחב:
\eq{
    \limn(a_n)=\infty
}
\qed

% 5.1
\setcounter{equation}{0}
\section{הוכיחו או הפריכו את הטענות הבאות:}
\sub{}
נרצה להוכיח כי:
\eq{
    \limn(a_n\cdot{b_n})=\infty
}
לכן, עלינו להוכיח:
\eq{
    \forall{M>0}~~\exists{N\in\N}~~\forall{n>N}~~a_n\cdot{b_n}>M
}
נתון כי $b_n$ חסומה מלרע החל ממקום מסוים ע"י מספר חיובי, כלומר:
\eq{
    \exists{m\in\mathbb{R}}~~\exists{N_1\in\N}~~\forall{n>N_1}~~b_n\geq{m}
}
בנוסף, נתון כי $a_n$ שואפת לאינסוף, לכן לפי הגדרה:
\eqn{
    \forall{M>0}~~\exists{N_2\in\N}~~\forall{n>N_2}~~a_n>M
}
לפי )1(, נוכל לבחור את $a_n$ להיות גדול יותר בפרט מ:
\eq{
    a_n>\frac{M}{m}
}
נבחר את $N$ להיות:
\eq{
    N=\max(N_1,N_2)
}
לכן, החל מהאינדקס ה-$N$'י בכל סדרה מתקיים לכל $n>N$:
\eq{
    a_n&>\frac{M}{m}\\
    b_n&\geq{m}
}
לפי אקסיומות הסדר, ניתן להסיק כי:
\eq{
    a_n\cdot{b_n}&>\frac{M}{m}\cdot{m}\\
    a_n\cdot{b_n}&>M
}
לפיכך, לפי הגדרת הגבול במובנו הרחב:
\eq{
    \limn(a_n\cdot{b_n})=\infty
}
\qed\pagebreak
% 5.2
\setcounter{equation}{0}
\sub{}
נרצה להוכיח כי:
\eq{
    \limn(a_n\cdot{b_n})=\infty
}
לכן, עלינו להוכיח:
\eq{
    \forall{M>0}~~\exists{N\in\N}~~\forall{n>N}~~a_n\cdot{b_n}>M
}
נתון כי $b_n$ מתכנסת ל-$L$, כלומר:
\eq{
    \forall\epsilon>0~~\exists{N_1}\in\N~~\forall{n}>N_1~~|b_n-L|<\epsilon
}
ובפרט:
\eqn{
    b_n>L-\epsilon
}
בנוסף, נתון כי $a_n$ שואפת לאינסוף, לכן לפי הגדרה:
\eqn{
    \forall{M>0}~~\exists{N_2\in\N}~~\forall{n>N_2}~~a_n>M
}
לפי )2(, נוכל לבחור את $a_n$ להיות גדול יותר בפרט מ:
\eq{
    a_n>\frac{M}{L-\epsilon}
}
נבחר את $N$ להיות:
\eq{
    N=\max(N_1,N_2)
}
לכן, החל מהאינדקס ה-$N$'י בכל סדרה מתקיים לכל $n>N$:
\eq{
    a_n&>\frac{M}{L-\epsilon}\\
    b_n&>{L-\epsilon}
}
לפי אקסיומות הסדר, ניתן להסיק כי:
\eq{
    a_n\cdot{b_n}&>\frac{M}{L-\epsilon}\cdot{L-\epsilon}\\
    a_n\cdot{b_n}&>M
}
לפיכך, לפי הגדרת הגבול במובנו הרחב:
\eq{
    \limn(a_n\cdot{b_n})=\infty
}
\qed\pagebreak
% 5.3
\setcounter{equation}{0}
\sub{}
נראה כי הטענה אינה נכונה.\\
ניקח את:
\eq{
    a_n&=n\\
    b_n&=\frac{(-1)^n}{n}
}
אמנם:
\eq{
    \limn{a_n}&=\infty\\
    \limn{b_n}&=0
}
אך הסימן של $b_n$ לא נשאר קבוע.\\
כלומר, כאשר אנו כופלים את $a_nb_n$ אנו מקבלים גבול מהצורה:
\eq{
    \limn a_nb_n=\frac{\infty}{\pm0}
}
ומכיוון שהסימן של המכנה אינו קבוע, לא ניתן לומר כי הסדרה מתכנסת ל$\infty$ או ל-$-\infty$.
\qed

% 6
\pagebreak
\section{}
% 6א
\sub{}
נרצה להראות כי הסדרה מוגדרת.\\
החשש הוא שתחת השורש יהיה ביטוי שלילי, אולם מכיוון ש-$t\geq{-2}$, הביטוי תמיד יהיה מוגדר.\\
כעת, נרצה להראות שהסדרה \textbf{מונוטונית עולה}.\\
כלומר, נרצה להראות כי לכל $n$ טבעי מתקיים:
\eq{
    a_{n+1}\geq{a_n}
}
נראה זאת באמצעות אינדוקציה:
\subsub{בסיס האינדוקציה}
\eq{
    &a_1=t\\
    &a_2=\sqrt{2+a_1}=\sqrt{2+t}
}
נניח בשלילה כי:
\eq{
    a_1&\geq{a_2}\\
    t&\geq\sqrt{2+t}
}
מכיוון ש$t$ מוגדר להיות ערך:
\eq{
    -2\leq{t}\leq2
}
נוכל לבחור את $t$ להיות $-2$ ונקבל:
\eq{
    -2&\geq\sqrt{2-2}\\
    -2&\geq0
}
זוהי סתירה, ולכן בסיס האינדוקציה נכון.
\subsub{הנחת האינדוקציה}
נניח שהביטוי הבא נכון:
\eq{
    a_{n+1}\geq{a_n}
}
\pagebreak
\subsub{צעד האינדוקציה}
נרצה להוכיח שהביטוי הבא נכון:
\eq{
    a_{n+2}\geq{a_{n+1}}
}
ידוע לנו כי:
\eq{
    a_{n+2}=\sqrt{2+a_{n+1}}\geq\sqrt{2+a_n}=a_{n+1}
}
ניתן לבצע את מעבר זה לפי הנחת האינדוקציה.\\
לכן, הוכחנו כי הסדרה \textbf{מונוטונית עולה}.\\
כעת, נרצה להראות באינדוקציה כי הסדרה \textbf{חסומה מלעיל} ע"י $2$.\\
כלומר, כי לכל $n$ טבעי מתקיים:
\eq{
    a_n\leq{2}
}
\subsub{בסיס האינדוקציה}
לפי הגדרת סעיף א':
\eq{
    a_1&=t\\
    -2\leq{t}\leq2
}
\subsub{הנחת האינדוקציה}
נניח כי:
\eq{
    a_n\leq2
}
\subsub{צעד האינדוקציה}
נרצה להראות כי מתקיים:
\eq{
    a_{n+1}\leq2
}
לפי כלל הנסיגה של הסדרה:
\eq{
    a_{n+1}=\sqrt{2+a_n}
}
ניקח את $a_n$ להיות הערך המקסימלי שלו, כלומר - $2$:
\eq{
    a_{n+1}=\sqrt{2+2}=\sqrt{4}=2\leq2
}
וכך, הוכחנו כי הסדרה \textbf{חסומה מלעיל} ע"י $2$.
% 6ב
\pagebreak
\sub{}
הסדרה \textbf{מוגדרת} לכל $n$ טבעי מכיוון ש-$t$ הוא חיובי.\\
לכן, אם רק נוסיף לו, או ניקח את השורש שלו, הביטוי תמיד יהיה מוגדר, מכיוון שתחת השורש לא יהיה ביטוי שלילי.\\
נראה באמצעות אינדוקציה כי הסדרה \textbf{מונוטונית יורדת}, כלומר כי לכל $n$ טבעי מתקיים:
\eq{
    a_{n+1}\leq{a_n}
}
\subsub{בסיס האינדוקציה}
לפי הגדרת סעיף ב':
\eq{
    a_1&=t>2\\
    a_2&=\sqrt{2+t}
}
לכן, נרצה להראות כי מתקיים:
\eq{
    \sqrt{2+t}\leq{t}
}
מכיוון ש$t>2$, ניקח $\epsilon>0$ כך ש-$2+\epsilon=t$.\\
מכיוון ש$\epsilon$ חיובי:
\eq{
    4+\epsilon<4+\epsilon+3\epsilon+\epsilon^2
}
לכן, ע"י לקיחת שורש לשני הצדדים החיוביים בהכרח, נוכל להראות כי:
\eq{
    \sqrt{2+t}=\sqrt{4+\epsilon}<2+\epsilon=t
}
כפי שרצינו להראות.
\subsub{הנחת האינדוקציה}
נניח כי:
\eq{
    a_{n+1}\leq{a_n}
}
\subsub{צעד האינדוקציה}
נרצה להראות כי מתקיים:
\eq{
    a_{n+2}\leq{a_{n+1}}
}
לפי כלל הנסיגה של הסדרה:
\eq{
    a_{n+2}&=\sqrt{2+a_{n+1}}\\
    a_{n+1}&=\sqrt{2+a_n}
}
לפי הנחת האינדוקציה:
\eq{
    \sqrt{2+a_{n+1}}\leq\sqrt{2+a_n}
}
וכך הראינו שהסדרה היא \textbf{מונוטונית יורדת}.
\pagebreak\\
כעת, נרצה להראות כי הסדרה \textbf{חסומה מלרע}, כלומר כי קיים $m$ כך שלכל $n$ טבעי:
\eq{
    a_n\geq{m}
}
נבחר את $m=2$ ונראה כי הוא חסם מלרע ע"י אינדוקציה.
\subsub{בסיס האינדוקציה}
לפי הגדרת סעיף ב':
\eq{
    a_1&=t>2=m
}
\subsub{הנחת האינדוקציה}
נניח כי:
\eq{
    a_{n}\geq2
}
\subsub{צעד האינדוקציה}
נרצה להראות כי מתקיים:
\eq{
    a_{n+1}\geq2
}
נוסיף את 2 לשני הצדדים של אי-השווין של הנחת האינדוקציה:
\eq{
    2+a_{n}\geq4
}
וכעת ניקח את השורש השני של שני הצדדים:
\eq{
    \sqrt{2+a_n}\geq2
}
לפי כלל הנסיגה של הסדרה:
\eq{
    a_{n+1}=\sqrt{2+a_n}\geq2
}
לכן, הוכחנו כי הסדרה \textbf{חסומה מלרע} ע"י $m=2$.
% 6ג
\sub{}
בסעיף א' הראינו כי $a_n$ חסומה מלעיל ומונוטונית עולה.\\
בכיתה, הוכחנו משפט המראה כי סדרה כזו מתכנסת אל הסופרמום שלה, והוא במקרה זה - $2$.\\
מצד שני, בסעיף ב' הראינו כי $a_n$ חסומה מלרע ומונוטונית יורדת.\\
מאותה הסיבה, סדרה זו מתכנסת אל האינפימום שלה, וגם הוא $2$.\\
לכן, אנו יכולים להסיק מכך כי אין חשיבות לערכו של $t$, כל עוד הערך שלו גדול או שווה ל-$-2$, כלומר כך ש-$a_n$ תהיה מוגדרת.
\qed


% End
\end{document}