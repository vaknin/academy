% Preamble
\documentclass[a4paper, 12pt, leqno]{article}
\usepackage[margin=1in]{geometry} % Set margin
\usepackage{amssymb,amsmath,amsthm, amsfonts} % Math libraries
\usepackage{pdfpages} % Insert PDF pages

% Hebrew support
\usepackage[utf8x]{inputenc}
\usepackage{culmus}
\usepackage[english,hebrew]{babel}
\selectlanguage{hebrew}


% Custom commands
\newcommand{\sub}[1]{\subsection{\underline{#1}}}
\newcommand{\subsub}[1]{\subsubsection{\underline{#1}}}
\newcommand{\F}{\ensuremath{\mathbb{F}}}
\newcommand{\N}{\ensuremath{\mathbb{N}}}
\newcommand{\RR}{\ensuremath{\mathbb{R}}}
\newcommand{\Onef}{\ensuremath{1_{\F}}}
\newcommand{\Zerof}{\ensuremath{0_{\F}}}
\newcommand{\eqbcuz}[1]{\text{~$\stackrel{(#1)}{=}$~}}
\newcommand{\eq}[1]{\begin{align*}#1\end{align*}}
\newcommand{\eqn}[1]{\begin{align}#1\end{align}}
\newcommand{\set}[1]{\big{\{} #1 \big{\}}}
\newcommand{\bigset}[1]{\bigg{\{} #1 \bigg{\}}}
\newcommand{\limn}{\lim_{n\to\infty}}
\newcommand{\sumn}{\sum^{n}_{i=1}}

\renewcommand{\qed}{\hfill\(\qedsymbol\)}
\renewcommand{\leq}{\leqslant}
\renewcommand{\geq}{\geqslant}

% Begin Document %
\begin{document}

% Title Page
\begin{titlepage}
    \begin{center}
        \vspace*{4cm}
    
        {\fontsize{32pt}{32pt}\selectfont \textbf{אינפי 1}}
        
        \vspace{0.4cm}
        
        {\LARGE
        תרגיל 8}
    
        \vfill
            
        {
            \Large\textbf{אביב וקנין}
            \\
            \selectlanguage{english}316017128
        }
    \end{center}
\end{titlepage}

% 1
\section{}
\sub{}
נראה כי סדרת מרחקי הקטעים הסגורים $[a_n,b_n]$ מתכנסת ל-0.\\
אנו בעצם רוצים להראות כי מתקיים:
\eq{
    \forall\varepsilon>0~~\exists{N}\in\N~~\forall{n}>N~~|b_n-a_n|<\varepsilon
}
נבחר את $N$ כך ש-$N>\frac{1}{\sqrt{\varepsilon}}$, ולכן מתקיים:
\eq{
    |b_n-a_n|=\frac{1}{n^2}<\frac{1}{N^2}<\frac{1}{(1\backslash\sqrt{\varepsilon})^2}=\varepsilon
}
לכן, מרחקי הקטעים מתכנסים ל-$0$, וסדרת הקטעים מקיימת את תנאי הלמה של קנטור.
\sub{}
נקרא לסדרת הקטעים מסעיף א' $c_n$ ונגדיר אותה כ:
\eq{
    c_n=b_n-a_n
}
בנוסף, הראינו כי $c_n$ מתכנסת ל-0, כלומר:
\eq{
    \limn{c_n}=0
}
לכן, מאריתמטיקה של גבולות:
\eq{
    &0=\limn{c_n}=\limn(b_n-a_n)=\limn{b_n}-\limn{a_n}\\
    &\limn{a_n}=\limn{b_n}
}
\sub{}
אנו צריכים למצוא $N$ ספציפי עבורו מתקיים:
\eq{
    |a_n-L|<\varepsilon
}
מסעיף ב', ידוע כי מתקיים:
\eq{
    \limn{a_n}=\limn{b_n}=L
}
ולכן זה זהה ללמצוא $N$ עבור:
\eq{
    |a_n-b_n|=|b_n-a_n|<\varepsilon
}
ולכן, אם נציב $\varepsilon=\frac{1}{100}$ בנוסחה למציאת $\varepsilon$ אשר מצאנו בסעיף א', נקבל:
\eq{
    N>\frac{1}{\sqrt{\varepsilon}}=\frac{1}{\sqrt{\frac{1}{100}}}=10
}
דוגמה ל-$N$ כזה היא 11.\qed

% 2
\section{}
\sub{}
\eq{
    a_1=0 && a_3=\frac{1}{2} && a_5 = \frac{3}{4} && a_7 = \frac{7}{8} && a_9 = \frac{15}{16}\\\\
    a_2=0 && a_4=\frac{1}{4} && a_6 = \frac{3}{8} && a_8 = \frac{7}{16} && a_{10} = \frac{15}{32}
}
\sub{}
\eq{
    &a_{2n+1}=\frac{n-2}{n-1}\\
    &a_{2n}=\frac{\frac{2^n}{2}-1}{2^n}=\frac{2^n-2}{2^{n+1}}
}
\sub{}
לסדרה יש שני גבולות חלקיים.
\eq{
    &\limn{a_{2n+1}}=\limn{\frac{\frac{n-2}{n}}{\frac{n-1}{n}}}=\limn{\frac{1-\frac{2}{n}}{1-\frac{1}{n}}}=1\\
    &\limn{a_{2n}}=\limn{\frac{2^n-2}{2^{n+1}}}=\limn{\frac{\frac{2^n-2}{2^n}}{\frac{2^{n+1}}{2^n}}}=\limn{\frac{1-\frac{2}{2^n}}{2}}=\frac{1}{2}
}

% 3
\section{}
\sub{?}
\sub{}
דוגמה לסדרה מסוג זה היא:
\eq{
    a_n=\sqrt{n}
}
המרחק בין איברי הסדרה מתכנס לאפס, אולם יחד עם זאת, סדרה זו שואפת לאינסוף, ולכן אינה מתכנסת - כלומר מתבדרת.

% 5
\setcounter{section}{4}
\section{}
\sub{}
בכדי להראות כי $a_n$ ניתנת לסכימה, נראה כי הטור שלה מתכנס.
\eq{
    &\sumn\frac{2^i}{3^i+1}<\sumn\frac{2^i}{3^i}=\sumn(\frac{2}{3})^i\\
    &=2(1-(\frac{2}{3})^n)<2
}
הראינו שהטור של $a_n$ חסום מלעיל, וכעת נראה שהוא מונוטוני עולה.\\
\eq{
    \sum^{n+1}_{i=1}a_i=\sum^{n}_{i=1}a_i+\frac{2^{n+1}}{3^{n+1}+1}>\sum^{n}_{i=1}a_i
}
מכיוון שהטור של $a_n$ הוא סדרה מונוטונית עולה וגם חסומה מלעיל, ניתן להסיק כי הוא מתכנס, ולכן, $a_n$ ניתנת לסכימה.
\sub{}
$b_n$ לא ניתנת לסכימה.\\
ידוע לנו כי:
\eq{
    \limn{\frac{1}{2n}}&=\infty\\
    \forall{n}\in\N~~\frac{1}{2n}&\leq\frac{1}{2n-1}
}
לכן, לפי משפט הסנדוויץ' במובן הרחב, ניתן להסיק כי:
\eq{
    \limn{\frac{1}{2n-1}}=\infty
}
ומכיוון שהסדרה לא מתכנסת - היא לא ניתנת לסכימה.
\sub{}
הוכחנו בהרצאה כי:
\eqn{
    \limn\sqrt[n]{n}=1
}
לכן:
\eq{
    \limn{\frac{1}{\sqrt[n]{n}}}=\frac{\limn{1}}{\limn{\sqrt[n]{n}}}=\frac{1}{1}=1
}
כלומר, לפי אריתמטיקה של גבולות הראינו כי הסדרה $c_n$ מתכנסת ל-1.\\
הראינו כי אם סדרה ניתנת לסכימה אזי היא מתכנסת ל-0, ולכן אם היא אינה מתכנסת ל-0, היא אינה ניתנת לסכימה, כמו במקרה זה.
\pagebreak\sub{}
\setcounter{equation}{0}
מכיוון ש-$p_n$ היא סדרה של מספרים טבעיים, ניתן להניח כי:
\eq{
    -1\leq(-1)^{p_n}\leq1
}
ולכן:
\eq{
    g_n=-\frac{1}{n^2}\leq\frac{(-1)^{p_n}}{n^2}\leq\frac{1}{n^2}
}
אם נעביר אגפים, נקבל:
\eqn{
    0\leq\frac{(-1)^{p_n}+1}{n^2}=f_n\leq\frac{2}{n^2}=e_n
}
אם נראה כי $e_n$ ניתנת לסכימה, אזי גם נראה כי $f_n$ ניתנת לסכימה.\\
ראשית, נראה כי:
\eqn{
    \frac{2}{k^2-k}=\frac{2k-2(k-1)}{(k-1)k}=\frac{2}{k-1}-\frac{2}{k}
}
כעת, נראה כי הטור של $e_n$ מתכנס, לכל $2\leq{n}\in\N$ מתקיים:
\eq{
    &\sum^N_{n=2}\frac{2}{n^2}<\sum^N_{n=2}\frac{2}{n^2-n}\eqbcuz{2}\sum^N_{n=2}\frac{2}{n-1}-\frac{2}{n}=\\
    &=(\frac{2}{1}-\frac{2}{2})+(\frac{2}{2}-\frac{2}{3})+(\frac{2}{3}-\frac{2}{4})+...-\frac{2}{N-1})+(\frac{2}{N-1}-\frac{2}{N})=\\
    &=2-\frac{2}{N}\\
    &\sum^\infty_{n=2}\frac{2}{n^2-n}=2
}
לכן, ניתן להסיק כי:
\eq{
    \sum^\infty_{n=2}\frac{2}{n^2}<2
}
כלומר, הטור של $e_n$ חסום מלעיל, נראה כי הוא גם מונוטוני עולה:
\eq{
    \sum^{n+1}_{n=1}\frac{2}{n^2}=\sum^{n}_{n=1}\frac{2}{n^2}+\frac{2}{(n+1)^2}>\sum^{n}_{n=1}\frac{2}{n^2}
}
מכיוון שהטור של $e_n$ חסום מלעיל ומונוטוני עולה, הוא גם מתכנס.\\
מכיוון ש$e_n$ ניתנת לסכימה, הרי גם $f_n$ ניתנת לסכימה, מאחר וכל איברי הסדרות אי-שליליים לפי 1.
בנוסף, גם הסדרה $-\frac{1}{n^2}$ ניתנת לסכימה, מאחר ו:
\eq{
    \frac{1}{n^2}-(-\frac{1}{n^2})=\frac{2}{n^2}
}
בנוסף:
\eq{
    f_n=\frac{(-1)^{p_n}+1}{n^2}-\frac{1}{n^2}=\frac{(-1)^{p_n}}{n^2}
}
מכיוון שסכום סדרות ניתנות לסכימה יוצר סדרה ניתנת לסכימה, הרי קיבלנו כי:
\eq{
    d_n=\frac{(-1)^{p_n}}{n^2}
}
ניתנת לסכימה.\qed\setcounter{equation}{0}

% 6
\section{}
\sub{}
אי שוויון הממוצעים

% End
\end{document}