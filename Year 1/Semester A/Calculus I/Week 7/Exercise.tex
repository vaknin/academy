% Preamble
\documentclass[a4paper, 12pt, leqno]{article}
\usepackage[margin=1in]{geometry} % Set margin
\usepackage{amssymb,amsmath,amsthm, amsfonts} % Math libraries

% Hebrew support
\usepackage[utf8x]{inputenc}
\usepackage{culmus}
\usepackage[english,hebrew]{babel}
\selectlanguage{english}

% Custom commands
\newcommand{\sub}[1]{\subsection{\underline{#1}}}
\newcommand{\subsub}[1]{\subsubsection{\underline{#1}}}
\newcommand{\RR}{\mathbb{R}}
\newcommand{\F}{\ensuremath{\mathbb{F}}}
\newcommand{\N}{\ensuremath{\mathbb{N}}}
\newcommand{\Onef}{\ensuremath{1_{\F}}}
\newcommand{\Zerof}{\ensuremath{0_{\F}}}
\newcommand{\eqbcuz}[1]{\text{~$\stackrel{(#1)}{=}$~}}
\newcommand{\eq}[1]{\begin{align*}#1\end{align*}}
\newcommand{\eqn}[1]{\begin{align}#1\end{align}}
\newcommand{\set}[1]{\big{\{} #1 \big{\}}}
\newcommand{\bigset}[1]{\bigg{\{} #1 \bigg{\}}}
\renewcommand{\qed}{\hfill\(\qedsymbol\)}
\renewcommand{\leq}{\leqslant}
\renewcommand{\geq}{\geqslant}
\newcommand{\limn}{\lim_{n\to\infty}}

% Begin Document %
\begin{document}

% Title Page
\begin{titlepage}
    \begin{center}
        \vspace*{4cm}
    
        {\fontsize{35pt}{35pt}\selectfont \textbf{אינפי 1}}
        
        \vspace{0.4cm}
        
        {\LARGEתרגיל 7}
    
        \vfill
            
        {\Large\textbf{אביב וקנין}\\821710613}
    \end{center}
\end{titlepage}

% 1
\section{הוכיחו או הפריכו}
\sub{}
הטענה נכונה.
לפי אריתמטיקה של גבולות:
\eq{
    \limn{a_n+b_n}&=\limn{a_n}+\limn{b_n}\\
    \limn{a_n}+\limn{b_n}&=\infty+\limn{b_n}\\
    \infty+\limn{b_n}&=L\\
    \limn{b_n}&=L-\infty=-\infty
}
\sub{}
הטענה אינה נכונה.\\
לשם דוגמה, ניקח את $a_n$ להיות:
\eq{
    a_n=n
}
וניקח את $b_n$ להיות $n^2$ עבור מספרים אי-זוגיים ו-$n^3$ עבור מספרים זוגיים, כלומר:
\eq{
    b_n=n^2,n^3,n^2,n^3...
}
אכן מתקיים:
\eq{
    \limn{a_n}&=\infty\\
    \limn\frac{a_n}{b_n}&=0\in\RR
}
אולם, $b_n$ מתבדרת, ולכן לא מתכנסת ל-$\infty$ או ל-$-\infty$.
\qed

% 2
\pagebreak
\section{}
\sub{}
אנו רוצים להראות כי $A_n$ מתכנסת ל-$L$.\\
ראשית, נחלק את $A_n$ לסכום של שני איברים:
\eq{
    A_n=\frac{a_1+a_2+...+a_N}{n}+\frac{a_{N+1}+a_{N+2}+...+a_n}{n}
}
\subsub{הביטוי השמאלי}
מכיוון שנתון כי $a_n$ מתכנסת, הרי היא חסומה ע"י $M$ כלשהו, לכן, נבחר $n$ כך ש:
\eq{
    \frac{a_1+a_2+...+a_N}{n}<\frac{N_1\cdot{M}}{n}<\frac{L+\varepsilon}{2}
}
\subsub{הביטוי הימני}
מכיוון שנתון כי $a_n$ מתכנסת ל-$L$:
\eq{
    \forall\varepsilon>0~\exists{N}\in\N~~\forall{n}>N~~|a_n-L|<\frac{L+\varepsilon}{2}
}
כעת, נוכל להראות כי:
\eq{
    |A_n-L|=|\frac{a_1+a_2+...+a_N}{n}+\frac{a_{N+1}+a_{N+2}+...+a_n}{n}|<\frac{L+\varepsilon}{2}+\frac{L+\varepsilon}{2}-L=\varepsilon
}
לכן, מהגדרת הגבול, $A_n-L$ מתכנסת ל-$L$.
\sub{}
הטענה אינה נכונה.\\
ניקח את:
\eq{
    a_n=(-1)^n
}
כלומר, $a_n$ אינה מתכנסת.\\
בנוסף:
\eq{
    \sum^n_{j=1}a_j=0
}
לכן:
\eq{
    A_n=\frac{1}{n}\sum^n_{j=1}a_j=0\implies\limn{A_n}=0
}
ניתן לראות כי $a_n$ מתבדרת ו-$A_n$ מתכנסת.
\qed

% 3
\section{}
\sub{}
הטענה אינה נכונה.\\
לשם דוגמה, ניקח את:
\eq{
    a_n&=n^{-1}\\
    m_n&=2n-1
}
לכן, $a_n$ לא חסומה מלעיל, מכיוון שהיא שואפת בסירוגין ל-$-\infty,\infty$.\\
ואילו $a_{m_n}$ חסומה מלעיל ע"י $0$.
כלומר, סדרה אחת מתבדרת בעוד השנייה מתכנסת - סתירה.
\qed
\sub{}
לשם פשטות, נסתכל על הצורה הקונטרה-חיובית של המשפט, ונראה כי אם $a_n$ חסומה מלעיל, אז גם $a_{m_n}$ חסומה מלעיל.\\
ידוע לנו כי $m_n$ היא סדרה עולה ממש של אינדקסים, כלומר - מספרים טבעיים.\\
בנוסף, ידוע כי $a_n$ חסומה מלעיל, אז:
\eq{
    \exists{M}\in\RR~~\forall{n}\in\N~~a_n\leq{M}
}
לכן, אם נשלב את שתי המסקנות, נגלה כי לא משנה מהו ערך $m_n$, ערך הסדרה $a_n$ בנקודה זו בהכרח קטן מ-$M$.
כלומר - הסדרה חסומה מלעיל.
\qed

% 4
\section{}
\sub{}
נניח כי מתקיים:
\eq{
    K\neq{L}
}
כלומר, הגבולות החלקיים של $a_n$ מתכנסים למספרים שונים, בהתאים לזוגיות או אי-זוגיות האינדקס.\\
לפיכך, אם תתי-הסדרות של $a_n$ מתכנסות לערכים שונים, הרי שהסדרה לא יכולה להתכנס לערך יחיד, וזה נובע באופן מיידי ממשפט הירושה.\\
כלומר - $a_n$ מתבדרת.
\sub{}
נניח כי מתקיים:
\eq{
    K={L}
}
לכן, בדומה לסעיף הקודם, אנו יודעים שהסדרה מקבלת גבול זהה בין אם האינדקס זוגי, ובין אם הוא אי-זוגי.\\
כלומר, לכל אינדקס, כאשר הסדרה $a_n$ שואפת לאינסוף, גבול הסדרה הוא $L$.
\qed

% 5
\pagebreak
\section{}
\sub{}
לפי הגדרת הסדרה האינדוקטיבית:
\eqn{
    b_{n+1}&=\frac{1}{1+b_n}\\
    b_{n+2}&=\frac{1}{1+b_{n+1}}
}
לפי )1( ו-)2( וקצת אלגברה:
\eq{
    b_{n+2}=\frac{1+b_n}{2+b_n}=1-\frac{1}{b_n+2}
}
\sub{}
נראה בעזרת אינדוקציה כי $b_{2k-1}=c_k$ היא סדרה מונוטונית עולה, כלומר כי מתקיים:
\eq{
    c_{k+1}\geq{c_k}
}
\subsub{בסיס האינדוקציה}
\eq{
    c_1&=b_{1}=0\\
    c_2&=b_{3}=\frac{1}{2}\\
    \frac{1}{2}&\geq0
}
\subsub{הנחת האינדוקציה}
נניח כי מתקיים $c_{k+1}\geq{c_k}$.
\subsub{צעד האינדוקציה}
נרצה להראות כי מתקיים:
\eq{
    c_{k+2}\geq{c_{k+1}}
}
לפי הגדרת הסדרה $c_k$, ידוע כי מתקיים:
\eq{
    c_{k}&=b_{2k-1}\\
    c_{k+1}&=b_{2k+1}\\
    c_{k+2}&=b_{2k+3}
}
בדומה לסעיף א', נבטא את $c_{k+1}$ בעזרת $c_{k}$ ואת $c_{k+2}$ בעזרת $c_{k+1}$:
\eqn{
    c_{k+1}&=\frac{1}{1+\frac{1}{1+c_k}}\\
    c_{k+2}&=1-\frac{1}{c_k+2}
}
לפי הנחת האינדוקציה, )3( ו-)4(:
\eq{
    c_{k+2}\geq{c_{k+1}}
}
כלומר, הראינו כי הסדרה $b_{2k-1}$ מונוטונית.\\
%%%%%%%%%%%%אינדוקציה שנייה%%%%%%%%%%%%
נראה בעזרת אינדוקציה כי $b_{2k}=d_k$ היא סדרה מונוטונית יורדת, כלומר כי מתקיים:
\eq{
    d_{k+1}\leq{d_k}
}
\subsub{בסיס האינדוקציה}
\eq{
    d_1&=b_{2}=1\\
    d_2&=b_{4}=\frac{2}{3}\\
    \frac{2}{3}&\leq1
}
\subsub{הנחת האינדוקציה}
נניח כי מתקיים $d_{k+1}\leq{d_k}$.
\subsub{צעד האינדוקציה}
נרצה להראות כי מתקיים:
\eq{
    d_{k+2}\leq{d_{k+1}}
}
לפי הגדרת הסדרה $d_k$, ידוע כי מתקיים:
\eq{
    d_{k}&=b_{2k}\\
    d_{k+1}&=b_{2k+2}\\
    d_{k+2}&=b_{2k+4}
}
בדומה לסעיף א', נראה את $d_{k+1}$ בעזרת $d_{k}$ ואת $d_{k+2}$ בעזרת $d_{k+1}$:
\eqn{
    d_{k+1}&=\frac{1}{1+\frac{1}{1+d_k}}\\
    d_{k+2}&=\frac{1}{1+\frac{1}{1+d_{k+1}}}
}
לפי הנחת האינדוקציה, )5( ו-)6(:
\eq{
    d_{k+2}\leq{d_{k+1}}
}
כלומר, הראינו כי הסדרה $b_{2k}$ מונוטונית.
\qed\pagebreak
\sub{}
נראה באינדוקציה כי $c_k$ חסומה מלעיל ע"י 1, ומכך נוכל להסיק כי היא מתכנסת.
כלומר, אנו רוצים להראות כי:
\eq{
    \forall{k}\in\N~~c_k\leq1
}
\subsub{בסיס האינדוקציה}
\eq{
    c_1=b_1=0\leq1
}
\subsub{הנחת האינדוקציה}
נניח כי מתקיים:
\eq{
    c_k\leq1
}
\subsub{צעד האינדוקציה}
אנו רוצים להראות כי מתקיים:
\eq{
    c_{k+1}\leq1
}
לפי סעיף א', ניתן לבטא את $c_{k+1}$ כ:
\eq{
    c_{k+1}=1-\frac{1}{c_k+2}
}
מכיוון שהסדרה $c_k$ חסומה מלרע ע"י 0 כי היא מונוטונית עולה, הרי הביטוי במכנה בהכרח חיובי, ולכן:
\eq{
    1-\frac{1}{c_k+2}\leq1
}
כפי שרצינו להוכיח.\\
**לא הצלחתי להראות כי $d_k$ חסומה מלרע.

% 6
\pagebreak
\setcounter{equation}{0}
\section{}
\sub{$\impliedby$}
נניח כי $a_n$ לא חסומה מלעיל.\\
כלומר, קיימת סדרת אינדקסים כלשהי $m_n$ אשר מקיימת:
\eq{
    \limn{a_{m_n}}=\infty
}
\sub{$\implies$}
נניח בשלילה כי הסדרה $a_n$ חסומה מלעיל, כלומר מתקיים:
\eqn{
    \exists{M}\in\RR~~\forall{n}\in\N~~a_n\leq{M}
}
בנוסף, נניח כי קיימת תת סדרה $a_{n_k}$ השואפת לאינסוף, כלומר מתקיים:
\eqn{
    \forall{M'}\in\RR~~\exists{n}\in\N~~a_n>M'
}
נשים לב כי הביטויים )1( ו-)2( מנוגדים זה לזה מבחינה לוגית.\\
לכן, נסיק כי ההנחה שלנו הייתה שגויה, ו-$a_n$ לא חסומה מלעיל.
\qed

% 7
\section{}
לפי טענה 14.5 בהרצאה - תת סדרה של תת-סדרה היא תת-סדרה.\\
לפי הנתון בתרגיל, \textbf{לכל} תת-סדרה של $a_n$ קיימת תת-סדרה המתכנסת ל-$L$.\\
בנוסף, לפי משפט הירושה, אנו יודעים כי אם סדרה מתכנסת ל-$L$ אזי \textbf{כל} תת-סדרה שלה מתכנסת ל-$L$ גם כן.\\
לכן, ניתן להסיק כי:
\eq{
    \limn{a_n}=L
}
\qed


% End
\end{document}