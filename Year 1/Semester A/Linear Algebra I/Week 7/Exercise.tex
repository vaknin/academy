% Preamble
\documentclass[a4paper, 12pt, leqno]{article}
\usepackage[margin=1in]{geometry} % Set margin
\usepackage{amssymb,amsmath,amsthm, amsfonts} % Math libraries

% Hebrew support
\usepackage[utf8x]{inputenc}
\usepackage{culmus}
\usepackage[english,hebrew]{babel}
\selectlanguage{hebrew}

% Custom commands
\newcommand{\sub}[1]{\subsection{\underline{#1}}}
\newcommand{\subsub}[1]{\subsubsection{\underline{#1}}}
\newcommand{\RR}{\mathbb{R}}
\newcommand{\F}{\ensuremath{\mathbb{F}}}
\newcommand{\N}{\ensuremath{\mathbb{N}}}
\newcommand{\Onef}{\ensuremath{1_{\F}}}
\newcommand{\Zerof}{\ensuremath{0_{\F}}}
\newcommand{\eqbcuz}[1]{\text{~$\stackrel{(#1)}{=}$~}}
\newcommand{\eq}[1]{\begin{align*}#1\end{align*}}
\newcommand{\eqn}[1]{\begin{align}#1\end{align}}
\newcommand{\set}[1]{\big{\{} #1 \big{\}}}
\newcommand{\bigset}[1]{\bigg{\{} #1 \bigg{\}}}
\renewcommand{\qed}{\hfill\(\qedsymbol\)}
\renewcommand{\leq}{\leqslant}
\renewcommand{\geq}{\geqslant}
\newcommand{\limn}{\lim_{n\to\infty}}

% Begin Document %
\begin{document}

% Title Page
\begin{titlepage}
    \begin{center}
        \vspace*{4cm}
        {\fontsize{35pt}{35pt}\selectfont \textbf{אלגברה לינארית 1}}
        \vspace{0.4cm}

        {\LARGEתרגיל 7}
        \vfill

        {\Large\textbf{אביב וקנין}\\
        \selectlanguage{english}316017128}
    \end{center}
\end{titlepage}

% 4
\setcounter{section}{3}
\section{}
נמצא את כל תתי המרחב של $\RR$.\\
יהא $W$ תת-מרחב של $\RR$, נחלק למקרים:
\sub{$W=0$}
מקרה זה הוא הטריוואלי.\\
וקטור האפס מהווה מרחב וקטורי, ולכן הוא תת-מרחב של $\RR$.
\sub{$W\neq0$}
מכיוון ש$W$ הינו תת מרחב, אנו יודעים כי הוא אינו ריק.\\
לכן, נוכל להניח כי הוא מכיל לפחות איבר אחד, אשר נקרא לו $w$.\\
מהגדרת תת-המרחב, ידוע כי:
\eq{
    \forall\lambda\in\RR~~\lambda{w}\in{W}
}
קל לראות כי ביטוי זה מבטא בעצם את כל הקומבינציות הלינאריות האפשריות ב$\RR$, או ליתר דיוק, ניתן לראות כי:
\eq{
    Span(W)=R
}
מהגדרת הפרוס הלינארי, ניתן להסיק כי:
\eqn{
    W\subset\RR
}
בנוסף, מהיות $W$ תת-מרחב של $\RR$:
\eqn{
    \RR\subset{W}
}
בשל ההכלה הדו צדדית ב)1( ו-)2(, הראינו כי:
\eq{
    W=\RR
}
כלומר, $\RR$ הוא תת-מרחב של עצמו.\\
לכן, הראינו כי תתי-המרחב של $\RR$ הם $0$ ו-$\RR$ עצמו.
\qed\pagebreak

% 7
\setcounter{section}{6}
\section{?????????}

\pagebreak
% 14
\setcounter{section}{13}
\section{}
$S$ תלויה לינארית.\\
בכדי שהיא לא תהיה תלויה לינארית, אמור להיות לה פתרון יחיד למערכת ההומוגנית - והוא הפתרון הטריוויאלי.\\
אולם, אם נדרג את המערכת לפי שורות נקבל:
\eq{
    \text{rref}(S)=\begin{bmatrix}
        1 & 0 & -\frac{1}{3} & 0\\
        0 & 1 & \frac{2}{3} & 0 \\
        0 & 0 & 0 & 1 \\
        0 & 0 & 0 & 0 
    \end{bmatrix}
}
נוכל לראות בבירור כי למערכת יש משתנה חופשי, וזה מספיק בכדי לקבוע כי ל-$S$ יש פתרון לא טריוויאלי.\\
כלומר - $S$ תלויה לינארית.
\qed

% 20
\setcounter{section}{19}
\section{}
\sub{$\implies$}
נניח כי $ad-bc\neq0$.\\
לכן, אנו יודעים כי המטריצה הבאה הפיכה:
\eq{
    A=
    \begin{bmatrix}
        a & c\\ b & d
    \end{bmatrix}
}
מכיוון ש-$A$ הפיכה, אנו יודעים כי היא שקולה לפי שורות ל-$I_2$.\\
לכן, אנו יודעים כי הפתרון היחיד שלה למערכת ההומוגנית הוא הפתרון הטריוויאלי, ולכן,
ניתן להסיק כי המערכת בלתי-תלויה לינארית.
\sub{$\impliedby$}
באופן דומה, אם נניח כי הקבוצה בלתי תלויה לינארית, נוכל להסיק מכך שאם נצרף את שני הוקטורים תחת מערכת אחת:
\eq{
    A=
    \begin{bmatrix}
        a & c\\ b & d
    \end{bmatrix}
}
נקבל וקטור בלתי-תלוי לינארית, אשר שקול לפי שורות ל-I.\\
הוכחנו בעבר כי וקטור כזה הוא הפיך, ומהיותו הפיך, ניתן להסיק כי:
\eq{
    \det(A)\neq{0}\implies ad-bc\neq0
}
\qed

% 22
\pagebreak
\setcounter{section}{21}
\section{}
מכיוון ש$S$ תלוי לינארית, נוכל לחסר את $v_3$ ממנו, וכך לקבל את $S'$, וכך:
\eq{
    Span(S)=Span(S')
}
ניקח את הוקטורים $S'=\set{v_1,v_2,v_4}$ ונדרג אותם:
\eq{
    \text{rref}(S')=
    \begin{bmatrix}
        1 & 0 & 0\\
        0 & 1 & 0\\
        0 & 0 & 1\\
        0 & 0 & 0
    \end{bmatrix}
}
כלומר, הראינו כי $S'$ היא קבוצה לא תלויה לינארית של וקטורים, ולכן, היא מהווה מרחב.
\qed

% 27
\setcounter{section}{26}
\section{}
\setcounter{equation}{0}
\sub{}
נראה כי $W_1,W_2$ הם תתי-מרחב של $V$.
\subsub{מטריצת האפס}
ראשית, אם נציב $a=b=c=0\in\F$ עבור שתי הקבוצות, נמצא את וקטור האפס.
\subsub{סגירות תחת כפל סקלרי וחיבור וקטורי}
נניח ש-$\vec{u},\vec{v}\in{W_1}$ וש-$t\in\F$.\\
בנוסף, נניח כי $\vec{x},\vec{y}\in{W_2}$.\\
לכן, אם נראה כי מתקיים:
\eq{
    t\vec{u}+\vec{v}&\in{W_1}\\
    t\vec{x}+\vec{y}&\in{W_2}
}
אזי $W_1,W_2\in{V}$.\\
נניח כי:
\eq{
    \vec{u}=
    \begin{bmatrix}
        a & -a\\
        b & c
    \end{bmatrix}
    &&
    \vec{v}=
    \begin{bmatrix}
        d & -d\\
        e & f
    \end{bmatrix}
    \\
    \vec{x}=
    \begin{bmatrix}
        a & b\\
        -a & c
    \end{bmatrix}
    &&
    \vec{y}=
    \begin{bmatrix}
        d & e\\
        -d & f
    \end{bmatrix}
}
לכן:
\eq{
    t\vec{u}+\vec{v}=c\cdot \begin{bmatrix}
        a & -a\\
        b & c
    \end{bmatrix}
    +
    \begin{bmatrix}
        d & -d\\
        e & f
    \end{bmatrix}
    =
    \begin{bmatrix}
        ta+d & -(ta+d)\\
        tb+e & tc+f
    \end{bmatrix}
}
קל לראות כי:
\eq{
    t\vec{u}+\vec{v}\in{W_1}
}
ולכן $W_1\leq{V}$.\\
בצורה דומה נראה עבור $W_2$:
\eq{
    t\vec{x}+\vec{y}=c\cdot \begin{bmatrix}
        a & b\\
        -a & c
    \end{bmatrix}
    +
    \begin{bmatrix}
        d & e\\
        -d & f
    \end{bmatrix}
    =
    \begin{bmatrix}
        ta+d & tb+e\\
        -(ta+d) & tc+f
    \end{bmatrix}
}
לכן, אנו רואים כי:
\eq{
    t\vec{x}+\vec{y}\in{W_2}
}
ולכן $W_2\leq{V}$.\qed
\sub{}
עלינו להראות כי:
\eq{
    W_1+W_2\leq{V}
}
\subsub{מטריצת האפס}
לפי א', אנו יודעים כי:
\eq{
    &0_{W_1}\in{W_1}\\
    &0_{W_2}\in{W_2}
}
ולכן:
\eq{
    0_{W_1}+0_{W_2}=0\in{W_1+W_2}
}
\subsub{סגירות תחת כפל סקלרי וחיבור וקטורי}
ניקח שני וקטורים השייכים לקבוצה $W_1+W_2$:
\eq{
    \vec{x},\vec{y}&\in{W_1+W_2}\\
    t&\in\F
}
אנו רוצים להראות כי:
\eq{
    t\vec{x}+\vec{y}\in{W_1+W_2}
}
בשל הגדרת הקבוצה $W_1+W_2$ אנו יודעים כי:
\eq{
    \vec{x}&=u_1+u_2\\
    \vec{y}&=v_1+v_2
}
כאשר:
\eq{
    u_1,v_1&\in{W_1}\\
    u_2,v_2&\in{W_2}
}
לכן מתקיים:
\eqn{
    t\vec{x}+\vec{y}&=t(u_1+u_2)+v_1+v_2\\
    &=(tu_1+v_1)+(tu_2+v_2)\nonumber
}
מכיוון שהראינו בסעיף א' כי הקבוצות הן תתי-מרחב, אנו יודעים כי הן סגורות תחת כפל סקלרי וחיבור וקטורי, ולכן:
\eqn{
    tu_1+v_1&\in{W_1}\\
    tu_2+v_2&\in{W_2}\nonumber
}
לכן, לפי )1( ו-)2(, אנו רואים כי כל אחד מהאיברים שייך לאחד מתתי-המרחב, ולכן הסכום שלהם שייך לסכום של תתי-המרחבים, כלומר $W_1+W_2$.\\
לכן, הראינו כי $W_1+W_2$ מהווה תת-מרחב וקטורי של $V$ מכיוון שהוא מכיל את מטריצת האפס והוא סגור תחת כפל וחיבור.
\qed
\sub{}



% End
\end{document}