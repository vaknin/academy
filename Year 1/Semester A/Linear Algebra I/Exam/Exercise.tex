% Preamble
\documentclass[a4paper, 12pt, leqno]{article}
\usepackage[margin=1in]{geometry} % Set margin
\usepackage{amssymb,amsmath,amsthm, amsfonts} % Math libraries

% Hebrew support
\usepackage[utf8x]{inputenc}
\usepackage{culmus}
\usepackage[english,hebrew]{babel}
\selectlanguage{hebrew}

% Custom commands
\newcommand{\sub}[1]{\subsection{\underline{#1}}}
\newcommand{\subsub}[1]{\subsubsection{\underline{#1}}}
\newcommand{\RR}{\mathbb{R}}
\newcommand{\F}{\ensuremath{\mathbb{F}}}
\newcommand{\N}{\ensuremath{\mathbb{N}}}
\newcommand{\Onef}{\ensuremath{1_{\F}}}
\newcommand{\Zerof}{\ensuremath{0_{\F}}}
\newcommand{\eqbcuz}[1]{\text{~$\stackrel{(#1)}{=}$~}}
\newcommand{\eq}[1]{\begin{align*}#1\end{align*}}
\newcommand{\eqn}[1]{\begin{align}#1\end{align}}
\newcommand{\set}[1]{\big{\{} #1 \big{\}}}
\newcommand{\bigset}[1]{\bigg{\{} #1 \bigg{\}}}
\renewcommand{\qed}{\hfill\(\qedsymbol\)}
\renewcommand{\leq}{\leqslant}
\renewcommand{\geq}{\geqslant}
\newcommand{\limn}{\lim_{n\to\infty}}

% Begin Document %
\begin{document}

% Title Page
\begin{titlepage}
    \begin{center}
        \vspace*{4cm}
        {\fontsize{35pt}{35pt}\selectfont \textbf{אלגברה לינארית 1}}
        \vspace{0.4cm}

        {\LARGEבוחן}
        \vfill

        {\Large\textbf{אביב וקנין}\\
        \selectlanguage{english}316017128}
    \end{center}
\end{titlepage}

% 1
\section{}
\sub{}
צירוף לינארי של $S$ הוא כל הקומבינציות האפשריות שניתן להגיע אליהן ע"י פעולות חיבור וקטורי וכפל בסקלר.\\
לאחר ביצוע פעולות אלו, אנו בהכרח נישאר במרחב הוקטורי $V$, והפעולות יהיו בהכרח מוגדרות בעזרת השדה $F$.
\sub{}
אם תת-הקבוצה $S$ היא תת-מרחב של $V$, אזי מתקיימים התנאים הבאים:
\begin{itemize}
    \item $S$ מכילה את וקטור האפס, או לחילופין - $S$ היא לא קבוצה ריקה
    \item $S$ סגורה תחת כפל סקלרי
    \item $S$ סגורה תחת חיבור וקטורי
\end{itemize}
\sub{}
ניתן לומר כי הקבוצה $S$ תלויה לינארית, אם אחד או יותר מהאיברים שלה, תלויים לינארית.\\ כלומר, אם ניתן להגיע מאיבר אחד לאיבר אחר, ע"י קומבינציה לינארית.\\
לדוגמה, נניח כי $s_1,s_2\in{S},~c\in\F$, אז דוגמה לתלות לינארית היא אם מתקיים:
\eq{
    s_1=c\cdot{s_2}
}
\qed\pagebreak

% 2
\section{}
נוכיח כי $W$ הינו מרחב וקטורי, לפי זה שהוא מקיים את התנאים ציינו בסעיף \textbf{1ב'}.
\sub{וקטור האפס}
מכיוון שכל תתי-המרחב הנמצאים בחיתוך מכילים את וקטור האפס בשל היותם תתי-מרחב, הרי גם החיתוך ביניהם מכיל את וקטור האפס.
\eq{
    0_v\in{W_i}~~\forall{i}~~1\leq{i}\leq15
}
\sub{סגירות לחיבור וקטורי ולכפל סקלרי}
מכיוון שעכשיו הוכחנו ש$0_v$ נמצא ב-$W$, אנו יודעים כי $W$ אינה קבוצה ריקה, לכן, נניח כי:
\eq{
    w_1,w_2\in{W},~~c\in\F
}
אנו יודעים כי $w_1,w_2$ נמצאים בחיתוך של כל תתי-המרחבים, ולכן, אנו יודעים כי הם אכן מקיימים את שני התנאים האחרים של תת-מרחב, כלומר, שני הוקטורים סגורים תחת חיבור וקטורי וכפל סקלרי.\\
לכן, אנו יכולים להסיק את הטענה הבאה:
\eq{
    w_1+c\cdot{w_2}\in{W}
}
לכן, מכיוון שהראינו שוקטור האפס נמצא ב-$W$ וגם כי האיברים בו סגורים תחת כפל סקלרי וחיבור וקטורי, ניתן להסיק כי $W$ הוא תת מרחב של $V$.
\qed\pagebreak

% 3
\section{}
כדי למצוא את המטריצות $P$ ו-$R$, נוסיף מימין למטריצה $A$ את מטריצת הזהות $I_3$ בתור מטריצה מורחבת, ולאחר מכן נדרג.
לבסוף, נקבל:
\eq{
    \begin{bmatrix}
    1 & 0 & 5 & 0 &|& -1 & 0 & 2\\
    0 & 1 & 2 & 0 &|& -2 & -1 & 1\\
    0 & 0 & 0 & 1 &|& 1 & 1 & 0
    \end{bmatrix}
}
לכן, המטריצה ההפיכה $P$ היא:
\eq{
    P=\begin{bmatrix}
    -1 & 0 & 2\\
    -2 & -1 & 1\\
    1 & 1 & 0
    \end{bmatrix}
}

והמטריצה בצורת מדרגות מצומצמת $R$ היא:
\eq{
    R=\begin{bmatrix}
        1 & 0 & 5 & 0\\
        0 & 1 & 2 & 0\\
        0 & 0 & 0 & 1
    \end{bmatrix}
}
\qed

% End
\end{document}