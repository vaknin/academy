% Preamble
\documentclass[a4paper, 12pt, leqno]{article}
\usepackage[margin=1in]{geometry} % Set margin
\usepackage{amssymb,amsmath,amsthm, amsfonts} % Math libraries

% Hebrew support
\usepackage[utf8x]{inputenc}
\usepackage{culmus}
\usepackage[english,hebrew]{babel}
\selectlanguage{hebrew}

% Custom commands
\newcommand{\sub}[1]{\subsection{\underline{#1}}}
\newcommand{\subsub}[1]{\subsubsection{\underline{#1}}}
\newcommand{\RR}{\mathbb{R}}
\newcommand{\F}{\ensuremath{\mathbb{F}}}
\newcommand{\N}{\ensuremath{\mathbb{N}}}
\newcommand{\Onef}{\ensuremath{1_{\F}}}
\newcommand{\Zerof}{\ensuremath{0_{\F}}}
\newcommand{\eqbcuz}[1]{\text{~$\stackrel{(#1)}{=}$~}}
\newcommand{\eq}[1]{\begin{align*}#1\end{align*}}
\newcommand{\eqn}[1]{\begin{align}#1\end{align}}
\newcommand{\set}[1]{\big{\{} #1 \big{\}}}
\newcommand{\bigset}[1]{\bigg{\{} #1 \bigg{\}}}
\renewcommand{\qed}{\hfill\(\qedsymbol\)}
\renewcommand{\leq}{\leqslant}
\renewcommand{\geq}{\geqslant}
\newcommand{\limn}{\lim_{n\to\infty}}

% Begin Document %
\begin{document}

% Title Page
\begin{titlepage}
    \begin{center}
        \vspace*{4cm}
        {\fontsize{35pt}{35pt}\selectfont \textbf{אלגברה לינארית 1}}
        \vspace{0.4cm}

        {\LARGEתרגיל 8}
        \vfill

        {\Large\textbf{אביב וקנין}\\
        \selectlanguage{english}316017128}
    \end{center}
\end{titlepage}

% 7
\setcounter{section}{6}
\section{}
\sub{$B_U$}
ראשית, נמצא בסיס ל-$U$ ע"י דירוג:
\eq{
    rref(U)=
    \begin{bmatrix}
        1&0&-1\\
        0&1&-2\\
        0&0&0\\
        0&0&0
    \end{bmatrix}
}
ולכן, הבסיס הוא:
\eq{
    B_U=\bigset{\begin{bmatrix}
        1\\
        0\\
        -1\\
        -2
    \end{bmatrix}
    ,\begin{bmatrix}
        -1\\
        -1\\
        0\\
        2
    \end{bmatrix}}
}
\sub{$B_W$}
שנית, נמצא בסיס ל-$W$ ע"י דירוג מטריצת הפתרונות ההומוגנית:
\eq{
    rref(W)=
    \begin{bmatrix}
        1&0&10&-7\\
        0&1&-3&2
    \end{bmatrix}
}
לכן, אם נבחר את $x_3=t$ ואת $x_4=k$, נקבל:
\eq{
    t\begin{bmatrix}
        -10\\3\\1\\0
    \end{bmatrix}
    +k\begin{bmatrix}
        7\\-2\\0\\1
    \end{bmatrix}
    =0_v
}
לכן:
\eq{
    B_W=\bigset{
        \begin{bmatrix}
            -10\\3\\1\\0
        \end{bmatrix}
        ,
        \begin{bmatrix}
            7\\-2\\0\\1
        \end{bmatrix}
    }
}
\sub{$\text{dim}(U+W)$}
כעת, כדי למצוא את המרחב של שני תתי-המרחבים, ניקח את שני הבסיסים, נשים אותם במטריצה $4\times4$ ונדרג, ונקבל:
\eq{
    rref(U+W)=
    \begin{bmatrix}
        1&0&0&*\\
        0&1&0&*\\
        0&0&1&*\\
        0&0&0&0
    \end{bmatrix}
}
ולכן, העמודה הרביעית תלויה לינארית, לכן הבסיס הוא:
\eq{
    B_{U+W}=\bigset{
        \begin{bmatrix}
            1\\0\\-1\\-2
        \end{bmatrix}
        ,
        \begin{bmatrix}
            -1\\-1\\0\\2
        \end{bmatrix}
        ,
        \begin{bmatrix}
            -10\\3\\1\\0
        \end{bmatrix}
    }
}
והמימד הוא:
\eq{
    \text{dim}(U+W)=3
}
\qed

% 9
\setcounter{section}{8}
\section{}
נבחר $l$ כלשהו.\\
יש לציין כי מכיוון ש-$L$ בלתי תלויה לינארית, היא בהכרח אינה מכילה את וקטור האפס, ולכן שני המקרים הבאים יכולים להתקיים.
\sub{$l$ הוא צירוף לינארי של איבר כלשהו מ-$G$}
במידה וזהו המקרה, נוכל "להחליף" את $g$ ב-$l$ ועדיין להשאר עם קבוצה אשר תיצור את $V$, וזאת מכיוון ששני האיברים תלויים לינארית, ולכן הפרוש הלינארי שלהם שווה.\\
לכן, במקרה זה, נקבל לפחות איבר אחד - $g$ אשר נוכל "להוציא" מהקבוצה היוצרת, ולהחליף ב-$l$.
\sub{$l$ הוא לא צירוף לינארי של איבר כלשהו מ-$G$}
מקרה זה הוא פשוט יותר, והוא מעיד על כך שנוכל לקחת כל איבר $g$ ולהחליפו ב-$l$ ועדיין לקבל קבוצה אשר תיצור את $V$.
\qed

% 10
\section{}
\sub{}
\subsub{פולינום האפס}
מכיוון ש-$U$ מכילה את כל האיברים מהצורה:
\eq{
    a+bz+{cz}^2
}
לכן, אם נציב $a=b=c=0$ נקבל:
\eq{
    0\in{U}
}
\subsub{סגירות תחת כפל וחיבור}
מכיוון שאנו יודעים שהקבוצה $U$ איננה ריקה, ניקח את שני הפולינומים $u_1,u_2\in{U}$ ואיבר שרירותי $c\in\mathbb{C}$, ונראה כי מתקיים:
\eq{
    cu_1+u_2\in\mathbb{U}
}

% 13
\setcounter{section}{12}
\section{}
תת המרחב הקטן ביותר הנמצא ב-$V$, או בעצם בכל תת-מרחב, הוא מרחב האפס, כלומר בהכרח מתקיים:
\eq{
    W_0=\set{0_v}
}
תת-המרחב הגדול ביותר הנמצא בכל מרחב, הוא המרחב עצמו, כלומר מתקיים:
\eq{
    W_m=V
}
אנו יודעים כי בין כל שני תתי-מרחבים בסדרת תתי-המרחבים, מתקיים:
\eq{
    \text{dim}(W_i+1)-\text{dim}(W_i)=1
}
וזאת מכיוון שכל תת-מרחב גדול בהכרח מקודמו, כי הרי האיבר הנוסף מהווה תוספת למימד, מכיוון שאם לא,
הרי יכולנו לבטא אותו באמצעות צירוף לינארי של תת-המרחב הקטן יותר.\\
אם לדוגמה ניקח מרחב וקטורי ממימד $1$, הרי סדרת תתי המרחבים תהיה:
\eq{
    &W_0=\set{0_v}\\
    &W_1=W_m=V
}
לכן, נוכל להסיק כי:
\eq{
    m=1=n
}
\qed

% 18
\setcounter{section}{17}
\section{}
לפי הנתון:
\eq{
    x_1(1,0,-1)+x_2(1,1,1)+x_3(1,0,0)=(a,b,c)
}
לכן:
\eq{
    &a=x_1+x_2+x_3\\
    &b=x_2\\
    &c=-x_1+x_2
}
לפי, לפי אריתמטיקה:
\eq{
    &x_1=b-c\\
    &x_2=b\\
    &x_3=a+c-2b
}
לכן:
\eq{
    [(a,b,c)]_B=(b-c,b,a+c-2b)
}
\qed

% 19
\section{}
\sub{}
נכניס את שני הוקטורים למטריצת שורות:
\eq{
    M=\begin{bmatrix}
        1&0&i\\
        1+i&1&-1
    \end{bmatrix}
}
ונדרג:
\eq{
    \text{rref}(M)=\begin{bmatrix}
        1&0&i\\
        0&1&-i
    \end{bmatrix}
}
מכיוון ששני הוקטורים לאחר דירוג יוצרים שתי שורות שאינן שורות אפסים, הם מהווים בסיס, ולכן $(v_1,v_2)$ מהווה בסיס סדור של $W$.
\sub{}
באופן דומה, נדרג את מטריצת השורות של $(w_1,w_2)$:

\eq{
    M=&\begin{bmatrix}
        1&1&0\\
        1&i&1+i
    \end{bmatrix}
    \\
    \text{rref}(M)=&\begin{bmatrix}
        1&0&i\\
        0&1&-i
    \end{bmatrix}
}
באופן דומה לסעיף א', שתי השורות אינן שורות אפסים ולכן הן מהוות בסיס סדור של $W$.
\sub{}
\subsub{$v_1$}
נחשב את $v_1$ ביחס לבסיס הסדור $(w_1,w_2)$:
\eq{
    x_1(1,1,0)+x_2(1,i,1+i)=(1,0,i)
}
ונקבל:
\eq{
    x_1&=\frac{1}{1+i}\\
    x_2&=\frac{i}{1+i}\\
}
ולכן, וקטורי הקואורדינטות של $v_1$ ביחס לבסיס הסדור $(w_1,w_2)$ הן:
\eq{
    (\frac{1}{1+i}, \frac{i}{1+i})
}
\subsub{$v_2$}
נחשב את $v_2$ ביחס לבסיס הסדור $(w_1,w_2)$:
\eq{
    x_1(1,1,0)+x_2(1,i,1+i)=(1+i,1,-1)
}
ונקבל:
\eq{
    x_1&=\frac{1+2i}{1+i}\\
    x_2&=\frac{-1}{1+i}\\
}
ולכן, וקטורי הקואורדינטות של $v_2$ ביחס לבסיס הסדור $(w_1,w_2)$ הן:
\eq{
    (\frac{1+2i}{1+i}, \frac{-1}{1+i})
}
\qed

% 21
\setcounter{section}{20}
\section{}
\sub{}
מכיוון ש-$B$ בלתי תלויה לינארית, הרי שלושת הוקטורים שהיא מכילה לא תלויים זה בזה.\\
ולכן, שלושת הוקטורים הללו בהכרח מהווים בסיס עבור $\text{span}(B)$, מיידית לפי ההגדרה.
\sub{}
נציג את $C$ כצירוף לינארי של $B$:
\eq{
    C=v_1[1,0,1]+v_2[1,1,1]+v_3[0,-1,1]
}
כעת, נוכל לדרג את שלושת מטריצות השורה שקיבלנו:
\eq{
    \text{rref}(
        \begin{bmatrix}
            1&0&1\\
            1&1&1\\
            0&-1&1
        \end{bmatrix}
    )
    =\begin{bmatrix}
        1&0&0\\
        0&1&0\\
        0&0&1
    \end{bmatrix}
}
והרי, קיבלנו את מטריצת הזהות.\\
כלומר, הראינו כי $C$ היא סדרה בלתי תלויה לינארית, ובדומה ל-$B$, היא מהווה בסיס סדור של $W$.
\sub{}
P צריכה להיות $3x3$

% End
\end{document}