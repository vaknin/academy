% Preamble
\documentclass[a4paper, 12pt, leqno]{article}
\usepackage[margin=1in]{geometry} % Set margin
\usepackage{amssymb,amsmath,amsthm, amsfonts} % Math libraries

% Hebrew support
\usepackage[utf8x]{inputenc}
\usepackage{culmus}
\usepackage[english,hebrew]{babel}
\selectlanguage{hebrew}

% Custom commands
\newcommand{\sub}[1]{\subsection{\underline{#1}}}
\newcommand{\subsub}[1]{\subsubsection{\underline{#1}}}
\newcommand{\F}{\ensuremath{\mathbb{F}}}
\newcommand{\N}{\ensuremath{\mathbb{N}}}
\newcommand{\Onef}{\ensuremath{1_{\F}}}
\newcommand{\Zerof}{\ensuremath{0_{\F}}}
\newcommand{\eqbcuz}[1]{\text{~$\stackrel{(#1)}{=}$~}}
\newcommand{\eq}[1]{\begin{align*}#1\end{align*}}
\newcommand{\eqn}[1]{\begin{align}#1\end{align}}
\newcommand{\set}[1]{\big{\{} #1 \big{\}}}
\newcommand{\bigset}[1]{\bigg{\{} #1 \bigg{\}}}
\renewcommand{\qed}{\hfill\(\qedsymbol\)}
\renewcommand{\leq}{\leqslant}
\renewcommand{\geq}{\geqslant}
\newcommand{\limn}{\lim_{n\to\infty}}

% Begin Document %
\begin{document}

% Title Page
\begin{titlepage}
    \begin{center}
        \vspace*{4cm}
        {\fontsize{35pt}{35pt}\selectfont \textbf{אלגברה לינארית 1}}
        \vspace{0.4cm}

        {\LARGEתרגיל 6}
        \vfill

        {\Large\textbf{אביב וקנין}\\
        \selectlanguage{english}316017128}
    \end{center}
\end{titlepage}

% 7
\setcounter{section}{6}
\section{נתון מרחב וקטורי $V$ מעל שדה $F$, הוכיחו או הפריכו את הטענות הבאות:}
\sub{}
הטענה נכונה.\\
ידוע כי:
\eq{
    a\cdot{u}+v=w
}
נניח כי קיים סקלר נוסף המקיים את הביטוי:
\eq{
    b\cdot{u}+v=w
}
נראה כי הם שווים זה לזה, וזה יעיד על יחידות הסקלר.
\eq{
    a\cdot{u}+v&=b\cdot{u}+v\\
    a\cdot{u}&=b\cdot{u}\\
    a&=b
}
אזי, קיים סקלר $a$ יחיד המקיים את הביטוי.
\sub{}
הטענה אינה נכונה.\\
ניקח את:
\eq{
    u=\begin{bmatrix}1\\0\end{bmatrix}&&
    v=\begin{bmatrix}0\\0\end{bmatrix}&&
    w=\begin{bmatrix}1\\1\end{bmatrix}&&
}
קל לראות כי לא קיים אף $a$ המקיים:
\eq{
    a\cdot{u}+v=w
}
\sub{}
הטענה נכונה.\\
לשם נוחות, נסתכל על הביטוי הקונטרה-חיובי:
\eq{
    a\neq0~~\land~~v\neq0\implies{av}\neq0
}
כעת, לפי הנתון:
\eq{
    a\neq0
}
לכן, נוכל לכפול את שני צדי אי-השוויון ב$v$:
\eq{
    av\neq0
}
ובצורה דומה, נוכל להוכיח עבור $v\neq0$.
\sub{}
הטענה נכונה.\\
ידוע כי:
\eq{
    av=v
}
לכן, נחסר $v$ משני צדי השוויון:
\eq{
    av-v&=v-v\\
    v(a-1)&=0
}
לכן, האופציה היחידה לקיום 0 היא אם $v=0$ או $a=1$.

% 14
\setcounter{section}{13}
\section{}
\setcounter{subsection}{3}
\sub{}
תת-הקבוצה $W$ אינה מהווה תת-מרחב וקטורי של $V$.\\
זאת מכיוון שהיא אינה סגורה לכפל בסקלר.\\
ראשית, נפשט את הביטוי:
\eq{
    W=\bigset{\begin{bmatrix}
        \frac{z}{2}\\2\\1
    \end{bmatrix}
    \big{|}~z\in\mathbb{R}
    }
}
כעת, נבחר את:
\eq{
    \lambda&=i\in\mathbb{C}\\
    y&=\begin{bmatrix}
        1\\4\\2
    \end{bmatrix}
    \in\mathbb{W}
}
ונראה כי:
\eq{
    \lambda\cdot{y}=\begin{bmatrix}
        i\\4i\\2i
    \end{bmatrix}
}
ואולם, בכדי ש$W$ תהיה סגורה לכפל בסקלר,  יש צורך ש:
\eq{
    \exists{z}\in\mathbb{R}~~z\cdot{y}=\begin{bmatrix}
        i\\4i\\2i
    \end{bmatrix}
}
וכמובן, דבר זה אינו אפשרי.
\sub{}
תת-הקבוצה $W$ אינה מהווה תת-מרחב וקטורי של $V$.\\
זאת מכיוון שהיא אינה סגורה לחיבור.\\
ניקח את:
\eq{
    u=\begin{bmatrix}
        1\\2\\2
    \end{bmatrix}
    \in\mathbb{W}
}
ונבצע חיבור של $u$ עם עצמו:
\eq{
    u+u=\begin{bmatrix}
        1\\2\\2
    \end{bmatrix}
    +
    \begin{bmatrix}
        1\\2\\2
    \end{bmatrix}
    =
    \begin{bmatrix}
        2\\4\\4
    \end{bmatrix}
    \not\in\mathbb{W}
}
כלומר, $u+v$ לא נמצא ב-$W$, כנדרש.

% 15
\section{}
\setcounter{subsection}{3}
\sub{}
פונקצייה זו אכן מהווה תת-מרחב.\\
ראשית, היא מכילה את פונקציית האפס:
\eq{
    0(-0)+0(0)=0
}
כעת, נראה סגירות לחיבור.\\

נניח ש-$f,g$ נמצאים בקבוצה.\\
כלומר, נרצה להראות כי מתקיים:
\eq{
    (f+g)(-x)+(f+g)(x)=0
}
נפשט את הביטוי:
\eq{
    (f+g)(-x)=f(-x)+g(-x)
}
לפי הגדרת הקבוצה:
\eq{
    f(-x)=-f(x)
}
לכן:
\eqn{
    f(-x)+g(-x)=-f(x)-g(x)
}
כעת, נעבוד על הביטוי השני:
\eqn{
    (f+g)(x)=f(x)+g(x)
}
נחבר את )1( ואת )2(:
\eq{
    -f(x)-g(x)+f(x)+g(x)&=\\
    &=-f(x)+f(x)-g(x)+g(x)=\\
    &=0
}
כנדרש.\\
כעת, נראה סגירות לכפל בסקלר.\\
נניח כי קיימת פונקציה $f$ השייכת לקבוצה, וסקלר $\lambda\in\mathbb{R}$.\\
נרצה להראות כי:
\eq{
    \lambda\cdot{f}(-x)+\lambda\cdot{f}(x)=0
}
נשתמש בהגדרת הקבוצה:
\eq{
    -\lambda{f}(x)+\lambda{f}(x)=0
}
לכן, הקבוצה היא תת-מרחב של $V$.
\sub{}
הפונקצייה אכן תת-מרחב.\\
פונקציית האפס נמצאת בה מכיוון שזוהי אכן הגדרת פונקציית האפס.\\
כעת, נראה סגירות לחיבור.\\
נניח כי $f, g$ נמצאות בקבוצה:
\eq{
    (f+g)(x)=f(x)+g(x)=0+0=0
}
כעת, נראה סגירות לכפל בסקלר.\\
נניח כי $\lambda\in\mathbb{R}$:
\eq{
    (\lambda{f})(x)=\lambda(f(x))=\lambda(0)=0
}
ולכן, $\lambda{f}$ בקבוצה, ולכן הקבוצה סגורה תחת כפל סקלרי.\\
מכאן, נוכל להניח כי הקבוצה היא אכן תת-מרחב של $V$.
\sub{}
הפונקצייה אינה תת-מרחב.\\
זאת מכיוון שהיא אינה סגורה תחת חיבור.\\
נניח כי $f,g$ שייכים לקבוצה, לכן, נרצה להראות כי:
\eq{
    |(f+g)(x)|\leq{1}
}
אולם לפי אי-שוויון ברנולי:
\eq{
    |(f+g)(x)|&=\\
    &=|f(x)+g(x)|\\
    &\leq|f(x)|+|g(x)|\\
    &\leq{2}
}
לכן, ניתן להסיק כי הקבוצה אינה סגורה תחת חיבור, ולכן אינה מהווה תת-מרחב של $V$.

% 26
\setcounter{section}{25}
\section{}
אנו צריכים למצוא וקטור $v$ אשר יקיים:
\eq{
    \mathbb{R}\begin{bmatrix}1\\-1\end{bmatrix}+v=\mathbb{R}^2_{col}
}
כלומר, עבור $t\in\mathbb{R}$:
\eq{
    \begin{bmatrix}t\\-t\end{bmatrix}+v=\mathbb{R}^2_{col}
}
דוגמה לוקטור אשר יקיים את הביטוי היא:
\eq{
    \begin{bmatrix}t\\-t\end{bmatrix}+\begin{bmatrix}u\\\pi\end{bmatrix}=\mathbb{R}^2_{col}~~u\in\mathbb{R}
}
וקטור נוסף:
\eq{
    \begin{bmatrix}t\\-t\end{bmatrix}+\begin{bmatrix}42\\u\end{bmatrix}=\mathbb{R}^2_{col}~~u\in\mathbb{R}
}


\pagebreak
% 31
\setcounter{section}{30}
\section{}
ידוע לנו כי $\langle\set{v_1,v_2,v_3}\rangle$ הוא תת-מרחב, כלומר, הוא סגור לחיבור ולכפל סקלרי.\\
לכן, אם נמצא קומבינציה לינארית אשר מקיימת $c_1v_1+c_2v_2+c_3v_3=w$, אזי נוכל להסיק ש-$w$ שייך לתת-המרחב.\\
ואכן המשתנים הבאים:
\eq{
    c_1&=1\\
    c_2&=-1\\
    c_3&=1\\
}
יפיקו את המשוואה הבאה:
\eq{
    \begin{bmatrix}
        2\\-1\\3\\2
    \end{bmatrix}
    -
    \begin{bmatrix}
        -1\\1\\1\\-3
    \end{bmatrix}
    +
    \begin{bmatrix}
        1\\1\\9\\-5
    \end{bmatrix}
    =
    \begin{bmatrix}
        4\\-1\\11\\0
    \end{bmatrix}
    =w
}
לכן, נסיק כי:
\eq{
    w\in{\langle\set{v_1,v_2,v_3}\rangle}
}

% End
\end{document}