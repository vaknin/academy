% Preamble
\documentclass[a4paper, 12pt]{article}
\usepackage[margin=1in]{geometry} % Set margin
\usepackage{pdfpages} % Insert pdf pages
\usepackage{amssymb,amsmath,amsthm, amsfonts} % Math libraries

% Custom commands
\newcommand{\sub}[1]{\subsection{\underline{#1}}}
\newcommand{\subsub}[1]{\subsubsection{\underline{#1}}}
\newcommand{\R}{\ensuremath{\mathbb{R}}}
\newcommand{\F}{\ensuremath{\mathbb{F}}}
\newcommand{\N}{\ensuremath{\mathbb{N}}}
\newcommand{\Onef}{\ensuremath{1_{\F}}}
\newcommand{\Zerof}{\ensuremath{0_{\F}}}
\newcommand{\eqbcuz}[1]{\text{~$\stackrel{(#1)}{=}$~}}
\newcommand{\eq}[1]{\begin{align*}#1\end{align*}}
\newcommand{\eqn}[1]{\begin{align}#1\end{align}}
\newcommand{\set}[1]{\big{\{} #1 \big{\}}}
\renewcommand{\qed}{\hfill\(\qedsymbol\)}
\newtheorem{lemma}{Lemma}

% Begin Document %
\begin{document}

% Title Page
\begin{titlepage}
    \includepdf{title.pdf}
\end{titlepage}

% 1
\section{Create an injective and surjective function from $\N$ to the natural numbers that can be divided by 5}
We'll define:
\eq{
    &f:N\longrightarrow{M}\\
    &f(n)=5n
}
We'll define $f$'s inverse function:
\eq{
    &f^{-1}:M\longrightarrow{N}\\
    &f^{-1}(n)=\frac{n}{5}
}
Now, we need to show that $f$ is injective and surjective.\\
We'll do that by showing that:
\eq{
    &f\circ{f^{-1}}=id_M\\
    &f^{-1}\circ{f}=id_N
}
\eq{
    &f\circ{f^{-1}}(n)=f(f^{-1}(n))=f(\frac{n}{5})=5\cdot\frac{n}{5}=n\\
    &f^{-1}\circ{f}(n)=f^{-1}(f(n))=f^{-1}(5n)=\frac{5n}{5}=n
}
Therefore, we've shown that:
\eq{
    &f\circ{f^{-1}}=id_M\\
    &f^{-1}\circ{f}=id_N
}
\qed

\section{Create an injective and surjective function from $\N$ to $\N\times\set{1,2,3}$}
\eq{
    x
}


\end{document}