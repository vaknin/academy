% Preamble
\documentclass[a4paper, 12pt, leqno]{article}
\usepackage[margin=1in]{geometry} % Set margin
\usepackage{amssymb,amsmath,amsthm, amsfonts} % Math libraries
\usepackage{pdfpages} % Insert PDF pages

% Hebrew support
\usepackage[utf8x]{inputenc}
\usepackage{culmus}
\usepackage[english,hebrew]{babel}
\selectlanguage{hebrew}

% Custom commands
\newcommand{\sub}[1]{\subsection{\underline{#1}}}
\newcommand{\subsub}[1]{\subsubsection{\underline{#1}}}
\newcommand{\F}{\ensuremath{\mathbb{F}}}
\newcommand{\N}{\ensuremath{\mathbb{N}}}
\newcommand{\RR}{\ensuremath{\mathbb{R}}}
\newcommand{\Onef}{\ensuremath{1_{\F}}}
\newcommand{\Zerof}{\ensuremath{0_{\F}}}
\newcommand{\eqbcuz}[1]{\text{~$\stackrel{(#1)}{=}$~}}
\newcommand{\eq}[1]{\begin{align*}#1\end{align*}}
\newcommand{\eqn}[1]{\begin{align}#1\end{align}}
\newcommand{\set}[1]{\big{\{} #1 \big{\}}}
\newcommand{\bigset}[1]{\bigg{\{} #1 \bigg{\}}}
\renewcommand{\qed}{\hfill\(\qedsymbol\)}
\renewcommand{\leq}{\leqslant}
\renewcommand{\geq}{\geqslant}
\newcommand{\limn}{\lim_{n\to\infty}}

% Begin Document %
\begin{document}

% Title Page
\begin{titlepage}
    \begin{center}
        \vspace*{4cm}
    
        {\fontsize{32pt}{32pt}\selectfont \textbf{מתמטיקה דיסקרטית}}
        
        \vspace{0.4cm}
        
        {\LARGE
        תרגיל 7}
    
        \vfill
            
        {
            \Large\textbf{אביב וקנין}
            \\
            \selectlanguage{english}316017128
        }
    \end{center}
\end{titlepage}

% 1
\section{}
\sub{}
נציב את:
\eq{
    x&=2^3\\
    y&=(-7)
}
בנוסחת הבינום של ניוטון, ונקבל:
\eq{
    &(8-7)^n=(1)^n=1=\sum^n_{i=0}\binom{n}{i}{2^3}^{n-i}(-1)^i7^i=\\
    &=2^3n\binom{n}{0}-7\cdot2^{3(n-1)}\binom{n}{1}+...+(-1)^n7^n\binom{n}{n}
}
כלומר, לפי נוסחת הבינום, הזהות מתקיימת.
\qed
\sub{}
לכל $x\in\RR$ מתקיים:
\eq{
    (x+1)^n=\sum_{k=0}^n\binom{n}{k}x^k
}
כעת, נגזור את שני הצדדים:
\eq{
    n(x+1)^{n-1}=\sum_{k=0}^n\binom{n}{k}kx^{k-1}
}
ונציב $x=2$:
\eq{
    n(2+1)^{n-1}=n\cdot3^{n-1}=\sum_{k=0}^n\binom{n}{k}k\cdot2^{k-1}
}
לאחר שנכפול ב-$2$ את שני צדי המשוואה, נקבל את הזהות.
\qed

% 2
\section{}
לכל $x\in\RR$ מתקיים:
\eq{
    (x+1)^n=\sum_{k=0}^n\binom{n}{k}x^k
}
כעת, נגזור את שני הצדדים:
\eq{
    n(x+1)^{n-1}=\sum_{k=0}^n\binom{n}{k}kx^{k-1}
}
נגזור שוב ונקבל:
\eq{
    (n-1)n(x+1)^{n-2}=\sum_{k=0}^n\binom{n}{k}(k-1)kx^{k-2}
}
נציב $x=1$ ונקבל את הזהות המבוקשת.
\qed

% 3
\section{}
לכל $x\in\RR$ מתקיים:
\eq{
    (x+1)^{2n}=\sum_{k=0}^n\binom{2n}{k}x^k
}
נציב $x=1$ ונקבל:
\eq{
    (1+1)^{2n}=\sum_{k=0}^n\binom{2n}{k}
}
נחלק ב-$2$ ונקבל את הזהות המבוקשת:
\eq{
    2^{2n-1}=\sum_{k=0}^n\binom{2n}{2k}
}
\qed

% 4
\section{}
נציב בנוסחת ההכלה והדחה ונקבל:
\eq{
    28&=|A\cap{B}\cap{C}|+15+14+11-6-5-4\\
    |A\cap{B}\cap{C}|&=3
}

% 5
\section{}
ניתן להסתכל על שאלה זו כלמצוא את כל הפונקציות $f$ מ-$A\backslash\set{a}$ ל-$B$ כך שהתמונה של $f$ מכילה את $B\backslash\set{b}$.\\
נסמן:
\eq{
    C=\set{f:f:A\backslash\set{a}\longrightarrow{B}}
}
נניח כי:
\eq{
    B=\set{1,2,3,b}
}
נציב בנוסחת ההכלה והדחה ונקבל:
\eq{
    |C_1\cap{C_2}\cap{C_3}|&=3\cdot3^5-3\cdot2^5+1=634
}
קבוצת כלל הפונקציות מחושבת ע"י:
\eq{
    4^5=1024
}
נחסר את הקבוצות ונקבל את התוצאה המבוקשת:
\eq{
    1024-634=390
}

% 6
\section{}
בכדי לחשב את מספר הדרכים, נרצה לחסר את האפשרויות ה"רעות" מכלל האפשרויות.\\
כלל האפשרויות:
\eq{
    \binom{15}{6}=5005
}
מספר האפשרויות הרעות במתמטיקה ובמדמ"ח, כלומר בלי לבחור אותם:
\eq{
    |M|=|C|=\binom{9}{6}=84
}
מספר האפשרויות הרעות בפיזיקה:
\eq{
    |P|=\binom{12}{6}=924
}
כעת, נרצה לחשב את:
\eq{
    |A|-|M\cup{C}\cup{P}|
}
ונעשה זאת ע"י הצבה בנוסחת ההכלה והדחה:
\eq{
    |M\cup{C}\cup{P}|=84+84+924-1-1=1090
}
לכן נקבל:
\eq{
    |A|-|M\cup{C}\cup{P}|=5005-1090=3915
}
\qed


% End
\end{document}