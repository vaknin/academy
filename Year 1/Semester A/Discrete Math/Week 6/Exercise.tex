% Preamble
\documentclass[a4paper, 12pt, leqno]{article}
\usepackage[margin=1in]{geometry} % Set margin
\usepackage{amssymb,amsmath,amsthm, amsfonts} % Math libraries
\usepackage{pdfpages} % Insert PDF pages

% Hebrew support
\usepackage[utf8x]{inputenc}
\usepackage{culmus}
\usepackage[english,hebrew]{babel}
\selectlanguage{hebrew}

% Custom commands
\newcommand{\sub}[1]{\subsection{\underline{#1}}}
\newcommand{\subsub}[1]{\subsubsection{\underline{#1}}}
\newcommand{\F}{\ensuremath{\mathbb{F}}}
\newcommand{\N}{\ensuremath{\mathbb{N}}}
\newcommand{\RR}{\ensuremath{\mathbb{R}}}
\newcommand{\Onef}{\ensuremath{1_{\F}}}
\newcommand{\Zerof}{\ensuremath{0_{\F}}}
\newcommand{\eqbcuz}[1]{\text{~$\stackrel{(#1)}{=}$~}}
\newcommand{\eq}[1]{\begin{align*}#1\end{align*}}
\newcommand{\eqn}[1]{\begin{align}#1\end{align}}
\newcommand{\set}[1]{\big{\{} #1 \big{\}}}
\newcommand{\bigset}[1]{\bigg{\{} #1 \bigg{\}}}
\renewcommand{\qed}{\hfill\(\qedsymbol\)}
\renewcommand{\leq}{\leqslant}
\renewcommand{\geq}{\geqslant}
\newcommand{\limn}{\lim_{n\to\infty}}

% Begin Document %
\begin{document}

% Title Page
\begin{titlepage}
    \begin{center}
        \vspace*{4cm}
    
        {\fontsize{32pt}{32pt}\selectfont \textbf{מתמטיקה דיסקרטית 1}}
        
        \vspace{0.4cm}
        
        {\LARGE
        תרגיל 6}
    
        \vfill
            
        {
            \Large\textbf{אביב וקנין}
            \\
            \selectlanguage{english}316017128
        }
    \end{center}
\end{titlepage}

% 1
\section{}
\sub{}
ניקח את כל הקלפים שמכילים 7, יש לכך בדיוק אפשרות אחת, ונכפיל במספר הקלפים שנשארו בחפיסה, כלומר:
\eq{
    1\cdot1\cdot1\cdot1\cdot48=48
}
\sub{}
עלינו לבחור 5 מספרים מתוך 13 הקלפים בעלי צורת היהלום, כלומר:
\eq{
    \binom{13}{5}=1287
}
\sub{}
ראשית, אנו צריכים לבחור את שני הקלפים שמכילים את הספרה 7, לכך יש $\binom{4}{2}=6$ אפשרויות.\\
כעת, אנו צריכים לבחור 3 ספרות שונות לשלושת הקלפים הנותרים, ולכך יש $\binom{12}{3}=220$ אפשרויות.\\
לאחר שעשינו זאת, עלינו לבחור צורות לשלושת הקלפים, ולכך יש לנו 4 אפשרויות לכל קלף, לבסוף נקבל:
\eq{
    6*220*4^3=84480
}

% 2
\section{}
ראשית, נאמר כי ישנן $\binom{15}{3}$ אפשרויות לסמן 3 משבצות בלוח בסה"כ.
\sub{}
כדי למצוא את מספר האפשרויות, נחסר את מספר האפשרויות שבעמודה הימנית יהיו בדיוק 0 שיבוצים ממספר האפשרויות בסה"כ.\\
כך, נקבל כי יש:
\eq{
    \binom{15}{3}-\binom{10}{3}=335
}
\sub{}
לשם כך, אנו צריכים לחשב את האפשרות שתהיה בדיוק משבצת מסומנת אחת בכל עמודה.\\
לעמודה אחת מספר האפשרויות הוא $\binom{5}{1}$, ולכן מספר האפשרויות הכולל הוא:
\eq{
    {\binom{5}{1}}^3=125
}

% 3
\section{}
בכדי לפתור זאת, נבין ראשית כי יש לנו $\binom{16}{3}$ אפשרויות למתוח קווים בין הקודקודים.\\
בכדי למצוא את מספר המשולשים, נחסיר ממספר זה את מספר הקווים אשר לא יפיקו משולשים, כלומר כל המתיחות קווים של שלושה קודקודים שונים על אותה הצלע של הריבוע.\\
לבסוף נקבל:
\eq{
    \binom{16}{3}-4^2=544
}

% 4
\section{}
ראשית, נחשב את מספר האפשרויות הכולל לסדר את הכדורים:
\eq{
    \binom{10}{3,3,2,2}=25200
}
ממספר זה נרצה לחסר את האפשרויות הלא חוקיות.\\
אם יהיו שני כדורים ירוקים או צהובים בקצוות, נקבל:
\eq{
    \binom{8}{3,3,2}=560
}
אם יהיו שני כדורים כחולים או אדומים בקצוות, נקבל:
\eq{
    \binom{8}{3,2,2,1}=1680
}
נחסר ונקבל את מספר האפשרויות:
\eq{
    25200-560\cdot2-1680\cdot2=20720
}

% 5
\section{}
ראשית, נבחר את הכדורים בקצוות ובאמצע, לכך יש לנו $6$ אפשרויות.\\
כעת, נותרנו עם 8 מקומות לפזר 3 כדורים אדומים וכחולים, ושני כדורים צהובים, כלומר:
\eq{
    \binom{8}{2,3,3}=560
}
נכפול את התוצאה ב6 ונקבל את מספר האפשרויות:
\eq{
    560\cdot6=3360
}

% 6
\pagebreak
\section{}
מכיוון שאנו מתחילים עם שני כדורים בכל מגירה, ניתן להתעלם מהם, ולחסר אותם משאר המשתנים.\\
כלומר, כל המגירות שלנו ריקות, ואנו צריכים להכניס 6 כדורים שונים ל6 מגרות, כך שבמגרה העליונה יהיו לכל היותר 4 כדורים.\\
נחשב את מספר האפשרויות הכולל, ונחסר ממנו את האפשרויות ה"רעות", כלומר האפשרויות בהן במגירה העליונה יש 5 או 6 כדורים.
מספר האפשרויות הכולל לסדר 6 כדורים ב6 מגירות:
\eq{
    \binom{6+6-1}{6-1}=462
}
אפשרויות רעות:
\eq{
    \binom{6}{6}+\binom{5}{1}=6
}
כלומר, מספר האפשרויות שלנו הוא:
\eq{
    462-6=456
}

% 7
\section{}
ראשית, נניח ארבע כדורים שחורים בשורה.\\
כעת, בגלל ההגבלה, נתחום בין כל שני כדורים שחורים שני כדורים לבנים.\\
בסיום תהליך זה, סידרנו 4 שחורים ו-6 לבנים.\\
מכיוון שהקצוות צריכים להיות שונים, יש לנו שני מקרים שונים, שחור בצד ימין או שחור בצד שמאל.
\sub{שחור בצד שמאל}
נניח כדור לבן נוסף בקצה הימני, בכדי שהקצוות יהיו בצבעים שונים.\\
כעת, עלינו לחלק עוד 5 כדורים לבנים, ויש לנו ארבעה מקומות שונים להניח אותם, כלומר:
\eq{
    \binom{5+4-1}{4-1}=56
}
\sub{שחור בצד ימין}
בדומה למקרה הקודם
עלינו לחלק 5 כדורים לבנים, ויש לנו ארבעה מקומות שונים להניח אותם, כלומר:
\eq{
    \binom{5+4-1}{4-1}=56
}
נחבר את האפשרויות ונקבל:
\eq{
    56\cdot2=112
}

% 8
\section{}
נתחיל מההגבלות:
\eq{
    1+2+3+4+5+6+7=28
}
כעת, נותר לנו להוסיף עוד 2 לסכום.\\
כלומר, יש לנו את הסכום 2, לחלק בין 7 מספרים:
\eq{
    \binom{2+7-1}{7-1}=\binom{8}{6}=28
}

% End
\end{document}