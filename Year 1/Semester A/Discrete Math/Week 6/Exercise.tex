% Preamble
\documentclass[a4paper, 12pt, leqno]{article}
\usepackage[margin=1in]{geometry} % Set margin
\usepackage{amssymb,amsmath,amsthm, amsfonts} % Math libraries
\usepackage{pdfpages} % Insert PDF pages

% Hebrew support
\usepackage[utf8x]{inputenc}
\usepackage{culmus}
\usepackage[english,hebrew]{babel}
\selectlanguage{hebrew}

% Custom commands
\newcommand{\sub}[1]{\subsection{\underline{#1}}}
\newcommand{\subsub}[1]{\subsubsection{\underline{#1}}}
\newcommand{\F}{\ensuremath{\mathbb{F}}}
\newcommand{\N}{\ensuremath{\mathbb{N}}}
\newcommand{\RR}{\ensuremath{\mathbb{R}}}
\newcommand{\Onef}{\ensuremath{1_{\F}}}
\newcommand{\Zerof}{\ensuremath{0_{\F}}}
\newcommand{\eqbcuz}[1]{\text{~$\stackrel{(#1)}{=}$~}}
\newcommand{\eq}[1]{\begin{align*}#1\end{align*}}
\newcommand{\eqn}[1]{\begin{align}#1\end{align}}
\newcommand{\set}[1]{\big{\{} #1 \big{\}}}
\newcommand{\bigset}[1]{\bigg{\{} #1 \bigg{\}}}
\renewcommand{\qed}{\hfill\(\qedsymbol\)}
\renewcommand{\leq}{\leqslant}
\renewcommand{\geq}{\geqslant}
\newcommand{\limn}{\lim_{n\to\infty}}

% Begin Document %
\begin{document}

% Title Page
\begin{titlepage}
    \begin{center}
        \vspace*{4cm}
    
        {\fontsize{32pt}{32pt}\selectfont \textbf{מתמטיקה דיסקרטית 1}}
        
        \vspace{0.4cm}
        
        {\LARGE
        תרגיל 6}
    
        \vfill
            
        {
            \Large\textbf{אביב וקנין}
            \\
            \selectlanguage{english}316017128
        }
    \end{center}
\end{titlepage}

% 1
\section{}
\sub{}
ניקח את כל הקלפים שמכילים 7, יש לכך בדיוק אפשרות אחת, ונכפיל במספר הקלפים שנשארו בחפיסה, כלומר:
\eq{
    1\cdot1\cdot1\cdot1\cdot48=48
}
\sub{}
עלינו לבחור 5 מספרים מתוך 13 הקלפים בעלי צורת היהלום, כלומר:
\eq{
    \binom{13}{5}=1287
}
\sub{}
ראשית, אנו צריכים לבחור את שני הקלפים שמכילים את הספרה 7, לכך יש $\binom{4}{2}=6$ אפשרויות.\\
כעת, אנו צריכים לבחור 3 ספרות שונות לשלושת הקלפים הנותרים, ולכך יש $\binom{12}{3}=220$ אפשרויות.\\
לאחר שעשינו זאת, עלינו לבחור צורות לשלושת הקלפים, ולכך יש לנו 4 אפשרויות לכל קלף, לבסוף נקבל:
\eq{
    6*220*4^3=84480
}

% 2
\section{}
ראשית, נאמר כי ישנן $\binom{15}{3}$ אפשרויות לסמן 3 משבצות בלוח בסה"כ.
\sub{}
כדי למצוא את מספר האפשרויות, נחסר את מספר האפשרויות שבעמודה הימנית יהיו בדיוק 0 שיבוצים ממספר האפשרויות בסה"כ.\\
כך, נקבל כי יש:
\eq{
    \binom{15}{3}-\binom{10}{3}=335
}
\sub{}
לשם כך, אנו צריכים לחשב את האפשרות שתהיה בדיוק משבצת מסומנת אחת בכל עמודה.\\
לעמודה אחת מספר האפשרויות הוא $\binom{5}{1}$, ולכן מספר האפשרויות הכולל הוא:
\eq{
    {\binom{5}{1}}^3=125
}

% 3
\section{}
בכדי לפתור זאת, נבין ראשית כי יש לנו $\binom{16}{3}$ אפשרויות למתוח קווים בין הקודקודים.\\
בכדי למצוא את מספר המשולשים, נחסיר ממספר זה את מספר הקווים אשר לא יפיקו משולשים, כלומר כל המתיחות קווים של שלושה קודקודים שונים על אותה הצלע של הריבוע.\\
לבסוף נקבל:
\eq{
    \binom{16}{3}-4^2=544
}

% 4
\section{}
ראשית, נחשב את מספר האפשרויות הכולל לסדר את הכדורים:
\eq{
    \binom{10}{3,3,2,2}=25200
}
ממספר זה נרצה לחסר את האפשרויות הלא חוקיות.\\
אם יהיו שני כדורים ירוקים או צהובים בקצוות, נקבל:
\eq{
    \binom{8}{3,3,2}=560
}
אם יהיו שני כדורים כחולים או אדומים בקצוות, נקבל:
\eq{
    \binom{8}{3,2,2,1}=1680
}
נחסר ונקבל את מספר האפשרויות:
\eq{
    25200-560\cdot2-1680\cdot2=20720
}

% 5
\section{}

% End
\end{document}